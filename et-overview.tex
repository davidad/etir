\subsubsection{AI and Robotics}

% [Can we make this paragraph a vision statement rather than a historical statement?
% Something more from http://su-etherpad.com/tpspace-The-Real-New-Space ?]
%   [davidad: this is intended to be a historical orientation, but feel free to add a
%    third paragraph that is more visionary in tone.]

In the broadest sense, the fields of AI and robotics strive to make non-human systems able to perform human tasks%
%Is this too limiting a view of AI and robotics? -- Will they just be worker tools, or the
%beginnings of a future of extraordinary intelligence and physical capabilities?)
, or other difficult tasks that are helpful to humans. Robotics focuses on the necessary physical capabilities, while AI focuses on the necessary mental capabilities. At present, AI systems are capable of solving a plethora of individual problems, each of them far more swiftly and accurately than a human could; but no robot is capable of loading a dishwasher yet.
% I would not make this point, as Robots already wash our dishes today far better than
% humans do (including total disinfection), and don't do it at all the way we do (contained
% machines). Planes also don't fly like birds, but we might say they are far better for our
% purposes.
This phenomenon is represented by the term "narrow AI": each AI system today is typically helpless outside the situation it was designed for. These systems solve problems such as searching the Internet, routing FedEx packages, military logistics, or playing chess. However, it has been predicted that in the future, we wil develop what is known as "strong AI": an AI system that nears human levels of tolerating uncertainty, and can generate original solutions to problems hitherto unseen and unanticipated by the designers of the AI.

\subsubsection{Biotechnology}



\subsubsection{Nanotechnology and Materials Science}



\subsubsection{Information Technology and Networking}



\subsubsection{Neurotechnology and Medicine}



\subsubsection{Policy, Law, and Ethics}

We are not in a position to fully analyze the far-reaching consequences of near-term steps by space agencies and private space entrepreneurs, however we can set about to establish a  context for critical evaluation of our motivations and hesitations at the Singularity University during the 2010 GSP. This is a time to develop a culture of intense questioning and reflexivity, weighing up the pros and cons of social and education value, economic and political drivers, scientific benefits, risks, the battle for sovereignty between nation states; time readiness, and the lessons learnt from past human endeavors at each stage of mission  planning and undertaking. Particular attention must be paid to our obligations and restrictions, the differences of moral standing and the agents behind them; the intrinsic and instrumental values of global peoples and the core truths of our calling, in order to plan effectively and to garner new insights and knowledge for wider reaching solutions to terrestrial concerns so that we can leverage exponentially advancing technologies for the benefit of humanity, the planet and the future of our activities in space. We have a responsibility to boldly stay to serve legitimate interests as a peaceful, well-meaning people and the opportunity to explore new ways of seeing life, and space, from a whole new perspective. 

It is posited that if we are to actually improve standards of living on Earth through space by  generating economic opportunity; providing access to new resources (material, intellectual and so on) then we need to bring about the prospect of rapid technological development, the cross-fertilization of ideas, new visions and shared dreams for the benefit of all human-kind and this begins with the power of the questions we ask ourselves, and the declarations we make for our future generations. 

\subsubsection{Entrepreneurship}

