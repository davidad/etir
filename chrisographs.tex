\subsection{10-year Time Horizon}


\subsubsection{Cheap Spacecraft for Developing Nations}

\begin{description}
\item[Application:] Use cheap spacecraft as a way to enable developing countries to get involved in space.

\item[Problem:] Third world countries are cut off from technology and science in many ways, and often the general public in such countries don't understand what space is and Earth is just a small part of a larger universe.

\item[Opportunity:]
Imagine the headlines: Ethopia launches their own satellite.If we give developing countries the tools to build and launch their own {\emph useful} satellite at a low-cost (via increasing capabilities of satellites and dramatic reduction in cost), we can:
\begin{enumerate}
\item get 3rd-world countries more involved in space and give them a greater understanding of the world we live in
\item Show the world that space is opening up to all people, and becoming increasingly cheaper to get involved.
\end{enumerate}

\item[Enabling technologies:]
The increase in computational power and decrease in cost is enabling satellites to be built at fractions of the current cost.

\item[Who is doing it:]
\hfill\begin{enumerate}
\item Chris Boshuizen and a team of NASA Ames Research Center engineers launched two Nexus One cellphones to 30,000 feet in July 2010, with plans to launch them to orbit in 2011 \cite{McNally2010}.
\item Bob Twiggs, inventor of the CubeSat, is working on launching eight ``pocket CubeSats'' into orbit in 2011. The cubes are an eighth of the volume of standard CubeSats at roughly 5 centimeters, and are expected to cost less than \$25,000 each with full capabilities \cite{Patel2010}.
\end{enumerate}

\item[Time Scale:] The materials and parts to launch a satellite for less than \$10,000 and eventually \$1,000 is approaching over the next five to ten years \cite{Patel2010}.

\item[Significant Bottlenecks:]
The largest bottleneck isn't the technology. Rather, it's the ability to acquire philanthropic funds for developing countries. Since most countries in Africa don't have the money for clean water or sustainable energy, this idea would have to be sold to philanthropists as a means to educate people in a way that enables them to build their own economy and future space program.
\end{description}

\subsubsection{DIY Spacecraft}

\begin{description}
\item[Application:] Build a satellite architecture that is on the order of hundreds of dollars to launch. Allows for individuals to own and operate their own missions.

\item[Problem:] Creating the business case. Creating the communication system to control and send/receive data from the space crafts.

\item[Opportunity:] With current technology a satellite can be made to sizes within 10 cm on a side and costing the price of a personal computer. With the advancements in nanotechnology, computing, and AI/robotics these spacecrafts will continue to shrink exponentially in size. If we can launch 1000s of these in one rocket the cost per satellite becomes very low.

Each launch will release a swarm of satellites that can serve many purposes. First, because each satellite is identical, all that needs to change to change its mission is the software program, which can be uploaded from ground. This means individuals can own or rent a satellite or any part of the swarm to operate their individual mission, that they create of find through open source software sharing. Monitor global warming from your desktop live, create science fair or research projects, take your own ``pale blue dot'' picture. Google may decide to buy or rent an entire swarm and take Google Earth live giving users incredible benefits such as Instant disaster relief monitoring, The best ideas are the ones that haven't been though of yet.This idea WILL open the space frontier to everyone on Earth.

\item[Estimate of the Potential Benefit:] This technology will be the next best thing to actually putting people in space. It will allow them to see space through their own eyes, which could in turn help motivate the opening of the space frontier.

\item[Who is doing it:] NASA Ames team (Chris Boshuizen, Matthew Reyes, William Marshall), many university Cubesat groups.
\end{description}

\subsubsection{Open Source Cube Satellites}

\begin{description}
\item[Application:] Open source/open cube satellites for development (OSCS)

\item[Problem:] Developing countries more often than not have no direct access to satellites and little
influence on application development, such as remote sensing, communication, and science.

\item[Opportunity:] OSCS can provide a new platform for developing countries to improve data
analytics for development. On open source/open access model is apt to spark innovation in areas
such as remote sensing for pandemics, resource identification, communication, education,
engineering, etc.
%USE of NASA CUBESAT ANNONCEMENT  for 2011 possible.
\cite{cubesatlaunch}

\item[Exponential Technologies:]    Nanotechnology, Computing, and/or
Networks.

\item[Grand Challenges:] Getting There, Space Science

\item[Other Connected Ideas:]   Intelligent Space Operating System (ISOS)

\item[Estimate of the Potential Benefit:]  Significant benefit for developing countries if combined
with terrestrial communication systems. Financing can be achieved through Worldbank and
NGO’s with the benefit that loans/donations are transparent (Satellite/infrastructure).

\item[Who is doing it:]
\hfill\begin{enumerate}
\item NASA
\item ESA
\item JAXA
\item MIT
\end{enumerate}

\item[Time Scale:] Immediately, increasingly cheaper in future

\item[Convergence:] Nanotechology

\item[Significant Bottlenecks:] Cheap Access to Space
\end{description}

\subsubsection{Intelligent Space Operating System}
\label{idea:isos}
\begin{description}
\item[Application:] Intelligent Space Operating System (ISOS)

\item[Problem:] Robotics, AI and communication systems for space more often than not are designed
in isolated fashion. Hence devices and data transmission is mostly non-interoperable, isolated
with great redundancies.

\item[Opportunity:] An Intelligent Space Operating System provides an open ecosystem for AI,
communication systems and satellites, as well as robotics to tie in by means of standardized
API’s. Parts of such ISOS can be open-source and reconfigurable. Open access, open purpose, is
granted as much as possible. AN ISOS ideally encompasses all technology within space and
provides unique identifiers (IPs with encrypted authentication for safety/privacy). Similar to the
earthbound internet (of things) the ISOS provides connectivity, packet switching and
intermittent storage, smart communication processes and optimal routing, satellite access, swarm
satellite tie-in, spacecraft and extraterrestrial station communication, inter and intra-stellar
communications, earthbound internet access and geo-stational satellites, real-time as well as time-
buffered information transmission, and inter-robotic communication.

\item[Exponential Technologies:]    Nanotechnology (Microsatellites), Computing Networks, Laser
technology, Solar Panels/energy

\item[Grand Challenges:] Robotic Exploration/ Communication \& Information Networks

\item[Estimate of the Potential Benefit:]  Eradication of redundancies, standardization, smart
information transmission.

\item[Who is doing it:]
\hfill\begin{enumerate}
\item Google
\item NASA
\item ESA
\item JAXA
\end{enumerate}

\item[Time Scale:] Incremental growth during the next years, inclining growth with evolution of microsatellites, exponential growth as soon as cheap access is possible.

\item[Convergence:] nanotech

\item[Significant Bottlenecks:] cheap access to space, but not essential
\end{description}
 
    \subsubsection{Intelligent Environments for Colonization} 
    \label{i2e}
\begin{description}  \item[Application:] -
 
   \item[Problem:] Extraterrestrial habitats for human space exploration  must provide a high level of sensing capabilities to assure life  support. If humans are out to colonize space, long term needs for human,  animal and plant survival must be met.
  
   \item[Opportunity:] Smart environments are already emerging on earth.  Due to innovation in sensor networks, sensor sophistication and  miniaturization, an increasing amount of environmental conditions can be  assessed. Sophisticated data analysis techniques allow for better  modeling of conditions and actions to be undertaken. Smart environments  such as a planetary or orbital station can connect to other nodes within  the Intelligent Space Operating Systems (see \autoref{idea:isos}) for optimization of  the overall space system state.
 
   \item[Exponential Technologies:] Nanotechnology, Biotechnology,  Computing, Networks.
 
   \item[Grand Challenges:] Habitats, biospheres, networks and  communication.
 
   \item[Other Connected Ideas:]  ISOS (Intelligent Space Operating  Systems)
 
   \item[Estimate of the Potential Benefit:] Essential for survival and  colonization, interconnectivity with other habitation nodes allows for  overall system optimization.
 
   \item[Who is doing it:]
  \hfill\begin{enumerate} 
     \item NASA
     \item Living-Planit
     \item Smart cities  MIT
    \item Responsive environments group MIT
     \end{enumerate}
 
   \item[Time Scale:] 5-10 years.
 
   \item[Convergence:] Nanotech and exponential ICT. Sensor development  and miniaturization.
   \end{description}

\subsubsection{Microbial Fuel Cells}
\label{microbe-fuel}
\begin{description}
\item[Application:] The use of  microbial fuel cells to generate power within
the Life Support Systems and any  different energetic applications in
spaceflight  or planetary stations could be a suitable, renewable and
efficient way of produce removable  energy.
\item[Problem:] The energy use for  life support systems and other
applications  during spaceflight and space station are restricted to
functions previously planned. The use  of additional energy system
could help  in building successful new life support systems and closed
loop systems.
\item[Opportunity:] \gls{MFC}s are devices  that use bacteria as the catalysts to
oxidize  organic and inorganic matter and generate current \cite{Rao1976}
. Electrons produced by the  bacteria from these substrates are
transferred  to the anode (negative terminal) and flow to the cathode
(positive terminal) linked by a  conductive material containing a
resistor,  or operated under a load (i.e., producing electricity that
runs a device). The construction and  analysis of \gls{MFC}s requires
knowledge  of different scientific and engineering fields, ranging from
microbiology and electrochemistry to  materials and environmental
engineering.  For example, was observed that when a water-soluble
distyrylstilbene oligoelectrolyte  (DSSN+) was added into bioreactors,
these  molecules were preferentially accumulate within cell membranes
and were used as the electron  transport mediator increasing the \gls{MFC}s
performance \cite{Garner2010}. In addition, Yuan et al. reported a novel bioelectrochemical  system to effectively reduce organic pollutant with utilization of the energy derived from a
microbial fuel cell \cite{Yuan2010}. In such a system,  there is a synergetic effect
between the  electrochemical and photocatalytic oxidation processes.
Both systems show great potential to  produce energy and to be part of
a closed  loop system for space exploration.
\item[Exponential  Technologies:] Biotechnology, Synthetic biology.
\item[Grand Challenges:] Human  exploration (staying there, life support systems)
\item[Other Connected Ideas:] Closed  loop systems, bioconductors
\item[Estimate  of the Potential Benefit:] A better source of energy which can
be used in space propulsion and  activation of improved life support
systems.
\item[Who is doing it:] \hfill
\begin{enumerate}
\item Dr. Bruce E. Logan Hydrogen  Energy Center, 212 Sackett Building,
Penn State  University, University Park, Pennsylvania
\item Dr. Shi-Jie Yuan, Department of  Chemistry, University of Science \&
Technology of China, Hefei, 230026,  China
\item Dr. Korneel Rabaey, and Dr.  Jeffrey M. Foley, Advanced Water
Management  Centre, Gehrmann Building, The University of Queensland,
Brisbane, Queensland 4072, Australia
\end{enumerate}
\item[Time Scale:] 5 years
\item[Convergence:] Being in the area  of bioreactors, this technology could
benefit big  challenges in up-cycle and energy areas.
\item[Significant Bottlenecks:]  \gls{MFC} research is a
rapidly evolving field that lacks  established terminology and methods
for the  analysis of system performance. This makes it difficult for
researchers to compare devices on an  equivalent basis.
The most of \glspl{MFC} require large volumes to produce more energy which is
a problem in launch costs.
\end{description}

 
\subsubsection{Polyethelyne/Biopolymer Radiation Shield} 
\label{biopolymer}
\begin{description}  \item[Application:] Generation of hydrogen rich polymers to shield the  inside of shuttles/inflatable inhabitable units from ionizing particle  and high energy electromagnetic radiation
\item[Problem:] 
The Earth’s magnetosphere prevents  protons, electrons and heavier nuclei that are continuously ejected by  the Sun from reaching the soil. Also, Earth's atmosphere blocks some of  the visible and infrared electromagnetic radiation, almost all the  ultraviolet and all the X and gamma rays. While humans have evolved a  system of UV protection involving melanin, no shielding is naturally  provided from particle radiation.

Human  exploration and colonization, as well as orbiting astronauts and  satellites in the shorter term, will greatly depend on the advances in  radiation protection technology, since the planets/moons/asteroids that  we will be in reach of most probably will not have an inherent  magnetosphere that will protect us.

Especially dangerous are the flares and coronal mass ejections generated by the  Sun in an observed nonstationary Poisson process, i.e. random  distribution \cite{Aschwanden2010}. These particles have  sufficient energy to get through any shielding material currently used  in space shuttles/space suits as well as the Earth's magnetosphere and because they are  unpredictable they constitute a major threat. \cite{solarsystemexplore}

Importantly,  the use of improper shielding material might cause the generation of  secondary radiation, i.e. acceleration of a charged particle through  deflection of another charged particle, which increases the probability  of absorption from the organism \cite{NCRP}.
It is therefore vital that  complete ionizing radiation protection be established before any human  colonization missions. 
 
\item[Opportunity:] It has been  observed that shielding materials of low atomic number, such as  hydrogen, are more effective against cosmic radiation (high energy  particle radiation from outer space) than, say, aluminum, because they  reduce the production of secondary radiation particles \cite{Aschwanden2010}. This can be extended to other ionizing high energy  particle radiation brought by the Sun. Polyethylene, because of its  hydrogenous nature, has been extensively used for shielding of crew  quarters on board of the ISS. If this system were to be united to the  upcycling of the hydrogen used to power the hydrogen-based external  propulsion engine proposed in this ETIR report in the ``Getting There''  problem space, as well as the upcycling of waste water inherently  produced in the system, greater achievements might be introduced in the  travelling phase. In order to establish a shielding system for optimal  colonization of other planets, research in the realm of synthetic  biology might lead us to the engineering of microrgnisms that can  produce highly hydrogenous biopolymers or the duration of our stay.  
 
\item[Exponential Technologies:]  nanotechnology, biotechnology
 
\item[Grand Challenges:] Getting  There, Staying There
 
\item[Other Connected Ideas:]  Ability to prediction solar flares and coronal mass ejections ahead of  bursts would enable for preventive measures to be taken in time for  reinforcement of protection. 
 
\item[Estimate of the Potential  Benefit:] Complete radiation protection would enable for a better  quality of life on other planets
 
\item[Who is doing it:] 
\hfill\begin{enumerate}
\item Shavers Laboratories at NASA  Langley Research Center (Hampton) and NASA Johnson Space Center  (Houston)
\item Aschwanden laboratories at  Lockheed Martin Advanced Technology Center, Solar \& Astrophysics  Laboratory (Palo Alto, CA) and McTiernan laboratories at Space Sciences  Laboratory, University of California, Berkeley
\end{enumerate}
 
\item[Time Scale:] 20 years
 
\item[Convergence:] External propulsion engine, water upcycling on space shuttles and synthetic  biology for microbial engineering 
 
\item[Significant Bottlenecks:]  There is no unique solution. Different approaches must be taken to make  different legs of the journey possible. Ideally, there would be one  shielding material everything would be made of.
   \end{description}
 
 
\subsubsection{Protein Engineering for Drug Design}
\label{bio-research}
 
\begin{description}  \item[Application:] X-ray \gls{crystallography} and electron \gls{crystallography}  in space can be used to obtain higher quality 3D and 2D crystals,  respectively, of proteins engineered for specific drug target  interaction.
 
\item[Problem:] Protein \gls{crystallography} is a science in its own right. The process of making a  crystal is laborious and the protocol not automatable as it is slightly  different for each protein. Not only, the same method cannot be deployed  for soluble and insoluble (or membrane) proteins since the types of  crystals that they generate are different. On average only one over ten  proteins yields crystals suitable for structure determination \cite{Vergara2005}.
 
\item[Opportunity:] Increasing the  success rate of protein \gls{crystallography} using space.
 
\item[Exponential Technologies:]  X-ray \gls{crystallography}, Transmission Electron Microscope
 
\item[Grand Challenges:] Human exploration
 
\item[Other Connected Ideas:]  Having crystallisation facilities on board of a space shuttle or space  station will enable the analysis of any molecule whose atoms can be  arranged in solids. Asteroid composition as well as planet soil  compositions can be analysed this way, by sampling first, crystallizing  and then imaging.
 
\item[Estimate of the Potential  Benefit:]
 
\item[Who is doing it:]
\hfill\begin{enumerate}
\item Zagari laboratories at the  University of Naples ``Federico II''
\item  Dr. Julio Valdivia Silva's laboratories
\end{enumerate}
 
\item[Time Scale:] 5 years
 
\item[Convergence:] Advances in  X-ray and electron \gls{crystallography} technologies will make the production  of high quality crystals less fundamental for the generation of a high  quality image. Also, methods such as fiber diffraction, powder  diffraction and small-angle X-ray scattering are all technologies that  are deployed as an alternative to X-ray and electron \gls{crystallography} for  the production of lower resolution images, but will grow exponentially  over the next ten years and will become just as good options to X-ray  and electron \gls{crystallography}.
 
\item[Significant Bottlenecks:]  Crystal production, especially for membrane (insoluble) proteins.  Advances in X-ray and electron \gls{crystallography} technologies might result
   \end{description}
 
 
\subsubsection{Genome Sequencing/Microarrays for Detection of Cancer Progression}
 
\begin{description}  \item[Application:] Understanding the underlying principle by which ``apoptosis,'' or programmed cell death, is re-activated in cancer cells  in the absence of gravity (simulated microgravity) and use this  knowledge to tackle cancer progression on Earth. 
 
\item[Problem:] Cancer is a  relentless enemy that has yet to be defeated. One in every eight deaths  worldwide is caused by cancer \cite{AmericanCancerSociety2010}. Finding a  cure has proven challenging mostly because there is a basic lack of  understanding into the molecular mechanisms underlying the disease.
 
\item[Opportunity:] Culturing  cancer cells in simulated micro-gravity with the Rotating Wall Vessel  (RWV) has given us insights into the chance that an environment lacking  gravity might re-activate apoptosis mechanisms within cells \cite{Grimm}. ``Apoptosis'' is a term used in biology to indicate a mechanism  of programmed cell death that is essential to ensure the balanced turn  over of cells within healthy tissues. If switched off, as happens in  most cancer types, the cells are enabled to proliferate indefinitely,  eventually invading other tissues creating metastasis and causing the  death of the patient. Understanding the mechanisms by which  re-activation of apoptosis occurs in microgravity, we might be able to  apply these to cancer cells back on Earth and produce similar results.  In vivo results have been achieved in D. melanogaster (fruit fly)  experiments, in which a cohort of flies carrying spontaneous tumors was  flown into space for one generation (~20 days) and upon their return  showed a completely reverted phenotype, that is, the tumors were absent \cite{Fahlen2006}. 
 
\item[Exponential Technologies:]  Biotechnology

\item[Grand Challenges:] Gravity is one of the fundamental and universal physical parameters on Earth.  Therefore, one critical scientific question is how much are organisms  affected by changes in gravity and how does the answer to this question  affect human space exploration and colonization?  
 
\item[Other Connected Ideas:]  Heart Conditions, Parkinson's Disease, Alzheimer's Disease
 
\item[Estimate of the Potential  Benefit:] Possible insights into a cure to cancer
 
\item[Who is doing it:] 
\hfill\begin{enumerate}
\item Elizabeth Bladder  laboratories, Australian Centre for Astrobiology, University of New  South Wales, Sydney, Australia
\item  Augusto Coccoli's group in Germany at the University of Berlin
\end{enumerate}
 
\item[Time Scale:] 5 years
 
\item[Convergence:] AI/robotics to  ensure automatic manipulation of life support systems for in vivo  validation experiments in mice, which will be fundamental for the proof  of principle. Also, as nanotechnology hits the knee of the curve, the  fast and efficient delivery of vectors carrying a possible cure - such  as DNA sequences made to replace existing damaged ones - will be made  possible. An increase in the efficacy of simulated microgravity devices  will allow for experiments to be conducted with a better degree of  accuracy on Earth.  
 
\item[Significant Bottlenecks:]  Simulated microgravity is not a perfect model for zero gravity, as shown  in the studies by Julio Valdiva and others, in which the gene  expression pattern of the same cell culture in microgravity and space  showed poor correlation.
 
\end{description} 

\subsubsection{Space Standardization}
\label{ipip}
\begin{description}   \item[Application:] Create standards for software, hardware, and   communication in space.
 
\item[Problem:] The space  community is poised at a similar position to the position of computer   users when there were only a handful of computers in the world. Every space mission we make is different and independent. The physical   connectivity of the earliest networks such as \gls{ARPANET}, and the data connectivity enabled by standardisation of communication protocols,   enabled the internet technology explosion we still enjoy today. We are   in a similar position in space now, barely wetting our toes on the cosmic ocean, needing to begin to prepare for the voyage ahead.
 
\item[Opportunity:] The time has   come to develop hardware, software, and communications standards that   enable a distributed communications and functional network in space.   This will inspire new innovations in space (similar to creating the   internet protocols and letting the ``Youtubes'' and ``Amazons'' develop).  We would ideally not need to force anyone to adopt this model, but  rather  by simply connecting to this network and using the same  protocols the  user is granted huge benefits of not needing to develop  certain  components themselves.
 
This is not a business model any  more than \gls{Vint} Cerf's standardization of internet protocols was a   revenue source. However we could create specific businesses and release   them in tandem with this idea, showing the possibilities without   limiting the standardization structure to any given business model. 
In a sense this tandem approach comprises opening the floodgates and riding the wave out.
 
\item[Exponential Technologies:]   This can be viewed as an enabler of exponential technologies of the future. 
 
\item[Grand Challenges:] Getting There, Staying There, Human Exploration, Robotic Exploration

\item[Estimate of the Potential Benefit:] As humanity spreads into the solar system, this will be both   an extension of the internet (and just as indispensible), and a   framework for hardware and software that accelerates and innovates   growth. For example, we may start with swarm satellites around earth   that have standard positioning and inter-communications protocols. Then   we have swarm satellites around Mars (the ``flies''). Then we have many   autonomous builder robots that create human habitats on mars (the   ``ants''), and the fact that the ``ants'' and ``flies'' are distributed  systems that speak the same language and talk about their relative and   absolute positions, etc, in the same way, enables quick creation and   design iteration.
 
\item[Who is doing it:]
Vint Cerf is working with \gls{JPL} on   the Interplanetary Internet, and NASA has repurposed some of their  spacecraft for communication for more modern missions (1). However if   the early internet was forced to grow by repurposing, retrofitting, and   reverse engineering each computer whenever you add a new one, it would   not enable the explosive internet phenomenon we see today. This is   currently mission-by-mission and is not on a large scale.
 
\item[Time Scale:] The first  versions of these standards can be released in 1-2 years, in tandem  with  flexible pico-satellite businesses. However just like internet   protocols these standards need to be rigid enough to be useful, yet   organic enough to be able to adapt to changing needs.
 
\item[Convergence:] 
 
\item[Significant Bottlenecks:]   Monetary. Rather than creating the YouTube or the Amazon of space, we'd   be creating the Internet of space, upon which other successes ride.   Without tandem applications, this could be a huge bottleneck.
 
\cite{ipsnig}
\end{description}

 
\subsubsection{Swarm  Satellites}
\label{swarm}
 \begin{description}  \item[Application:] A Swarm of Satellites in \gls{LEO} provide a  ``plug-and-play'' platform for space-based applications.
 
  \item[Problem:] Providing  space-based services to earth requires today a huge investment in  infrastructure, and space-based services are therefore either very  expensive or restricted to those with very massive markets.
 
  \item[Opportunity:] A cloud of  micro-satellites positioned as a distributed, open platform for the  creation of space-based applications in radar imaging and communications can revolutionize the way we think about building space-based services  and applications. Initially in \gls{LEO}, a cloud like this could expand to  support our colonization and exploration efforts in the solar system.
 
  \item[Exponential Technologies:]  Computing, Networks, AI, Robotics
 
  \item[Grand Challenges:] Getting  There, Staying There
 
  \item[Other Connected Ideas:]  Swarms of robots on the moon, swarms of asteroid-prospecting /  asteroid-mining robots; nano/femtosatellites; satellite-on-a-phone;
 
  \item[Estimate of the Potential  Benefit:]New, low-cost, very targeted space-based services can affect  positively the lives of billions. The current Commercial  Satellites/Services market is a 160 Billion dollars a year market  (including mass-marketed services such as Satellite TV and GPS  services), the availability of a platform like the one we suggest has  the potential to revolutionize this market, helping it grow by an order  of magnitud. Also, we predict that the ability to create space-based  applications without fixed infrastructure cost will have the biggest  effect in developing countries/economies, both for communications and as  a tool to leverage agriculture, fishing, primary goods extraction and  processing, disaster monitoring and emergency response.
 
  \item[Who is doing it:]
  \hfill\begin{enumerate}
\item The \gls{Techsat21} Project attempted  to launch a spaceborne sparse array for radar, but was cancelled.
\item Cubesats are open-source  micro-satellites being built by universities and small companies.
\item DARPA is funding a ``Fractionated  Spacecraft'' program,  attempting to create a inter-communicating sparse array of spacecraft.  The second phase of the program was awarded to Orbital Sciences, along  with IBM and \gls{JPL}, in December 2009.
 \end{enumerate}
 
  \item[Time Scale:] This platform  can be built today.
 
  \item[Convergence:] This  technologies will be able to benefit in the future from advances in  Nano-Materials for energy harvesting and storage; also, the concept  would benefit greatly from cheap launch
 
  \item[Significant Bottlenecks:]
 \hfill\begin{enumerate}
\item Precise Positioning in orbit for  the purposes of Sparse Phased-Arrays for Radar requires the development  of new algorithms that leverage the collective of the swarm to increase  definition.
\item Space Debris is an important  problem, and although the micro-satellites will de-orbit naturally in  the course of a few years,  doubling the number of active satellites in  orbit requires significant study.
\end{enumerate}
\end{description}
 
 
\subsubsection{Supercomputers in Space }
\label{supercom}
\begin{description}  \item[Application:] Data centers and supercomputers in space stations.
\item[Problem:] Time lag between  Earth and the sent missions, too much data is lost because it has not  been captured the moment it happened, future missions to further planets  would be difficult if there's not a base station that can communicate,  receive and send information directly.
\item[Opportunity:] 
Lots of data can be gathered and  detected from space (Temperature variations, radiation, cosmic rays,  etc.). Different simulations can be done efficiently in space, and  direct communication can be achieved between future robotic exploration  missions and the base space stations.
\item[Grand  Challenges:] Staying there, human exploration, robot exploration
\item[Other Connected Ideas:] Computing, memory, programming,  processing power, strong AI, narrow AI
\item[Estimate  of the Potential Benefit:] 
\item[Who  is doing it:]
\item[Time Scale:] 10-20 years
\item[Convergence:] \gls{AGI}, Narrow  AI, Computation, Transistors, Quantum Computation, Memory, Solar Power,  processing power
\item[Significant Bottlenecks:]  Computation power, Launching, Cost, Maintenance, Politics, AI
\end{description}
 
   \subsubsection{CAD with Space Physics Simulation} 
 \begin{description}  \item[Application:] A space version of \gls{CAD}/multiphysics with space-physics simulation  (microgravity, radiation, insulation, vacuum sealing, etc.)
 
   \item[Problem:] Reduce product development cycles by reducing time to  design and test spacecraft, tools, technologies and habitat designs.
 
   \item[Opportunity:] Standardize a platform for rapid design and  prototyping for commercial space, research and education.
 
   \item[Exponential Technologies:] AI, Robotics, Computing and Networks.
 
   \item[Grand Challenges:] Getting There, Staying There
 
   \item[Other Connected Ideas:] 3D printing in-space
 
%  \item[Estimate of the Potential  Benefit:]-
 
   \item[Who is doing it:] To  some extent, Ansys and Autodesk, but a specific space engineering  product or profile does not exist today.
 
   \item[Time Scale:] <5 years

      \end{description}
  
\subsubsection{Radiator  / Nanotube Arranged in a ``Forest''}
\label{thermal-materials}
\begin{description}  \item[Application:] Generating electricity, or controlling temperature  in space can depend on the thermal properties of radiators.
 
\item[Problem:] Radiators can be  large, heavy, or their properities can be degraded by contamination.
 
\item[Opportunity:] Single-Walled Carbon Nanotubes (SWCNT), arranged in a ``forest'' have very high  absorptivity to light and other electromagnetic radiation, and very good  thermal conductive properties. This could reduce the size of or  increase the performance of existing radiators. Bucky paper  \cite{Wang2008, nanolab, Wardle2008}  represents a form of material currently in production that could serve  this role.
 
\item[Exponential Technologies:]  Nanotechnology
 
\item[Grand Challenges:] Robotic  Exploration, Staying There
 
\item[Other Connected Ideas:]  Structural nanomaterials, nanotubes, graphene
 
\item[Estimate of the Potential  Benefit:] As a thermal conductor: Graphitic materials such as nanotubes  have up to an order of magnitude better thermal conductivity than silver  (the metal with the highest thermal conductivity), allowing fast  dissipation of heat. As an absorber: Current black paints have  absorptivities of about 0.94 \cite{solaracfaq}, but SWCNT have absorptivity up to  0.99, which may be useful for example on baffles on the insides of space  telescopes to absorb stray light.
 
\item[Who is doing it:]
\hfill\begin{enumerate}
\item Scientists Harold Kroto from  the University of Sussex, and Robert Curl Jr. and Richard Smalley from  Rice University shared the 1996 Noble Prize in Chemistry for their work  on fullerenes.
\item The Florida Advanced Center  for Composite Technologies (FACCT or FAC2T) directed by Ben Wang,  Professor of Industrial Engineering at the Florida A\&M  University-FSU College of Engineering, have carried out pioneering  research on the properties, bulk manufacture, integration into  nanocomposites, and applications of
'buckypaper'  since the year 2000, when Dr. Wang was first introduced to it. \cite{solaracfaq}
\item Tsinghua-Foxconn  Nanotechnology Research Center and Department of Physics, Tsinghua  University, Beijing 100084, People's Republic of China  \cite{Wang2008}
\item Arguably the world's most  renoun expert on large-scale nanotube ribbons, formed from drawing and  weaving nanotubes, is Ray Baughman at the Nanotech Institute at  UT-Dallas. He pioneered a technique to make carbon nanotube sheets and  yarns \cite{nanotech}
\item However carbon nanotubes are  fragile and often must be embedded in a polymer or ceramic matrix.  Pioneers of nanotube-embedded ceramics include Robert Davis  \cite{Suhr2008} and  Pulickel Ajayan  \cite{Hutchison2010}, and Eric A. Verploegen for nanotube-embedded  polymers.
\end{enumerate}

\item[Time Scale:] These materials  have been produced in the laboratory for several years, and are even  available for purchase \cite{nanolab}. It appears that they may be ready for a  demonstration application in the near-term.
 
\item[Convergence:] Being in the  area of nano-materials, this technology could benefit from advances in  nano-scale assembly.
 
\item[Significant Bottlenecks:]  Bare forests or woven nanotubes (such as those made by Ray Baughman) are  very fragile. Also, making good thermal contact to the nanotubes  appears to be the greatest challenge in wicking heat away from desired  areas.
 \cite{buckypapernano}
 \end{description}

\subsubsection{Psychological Well Being for Humans in Space: Neurofeedback}
\begin{description}   \item[Application:] Neurofeedback for psychological well-being
 
\item[Problem:] Stress,   deprivation, isolation, and confinement constitute major risks for the   physical and mental health of space explorers during the missions. However, limited resources are available for psychological   treatments in space.
 
\item[Opportunity:] The   affordability and use of tools such as \gls{fMRI} and \gls{EEG} elucidate the spatial   and temporal elements of neuronal activity. The increase in graphical   computational power and the improvement in 3D perception devices make  the participants immerse more in a \gls{VR} environment.
 
\item[Exponential Technologies:]   Neural Technologies and Medicine and Computing and Networking
 
\item[Grand Challenges:] Staying There, Space Exploration
 
\item[Other Connected Ideas:]   Cyberspace Worlds
 
\item[Estimate of Potential   Benefit:] Utilizing the real-time noninvasive neural signals recorded,   neurofeedback and virtual reality will help the subjects to learn how  to  control over their own brain activation \cite{deCharms2008}.  Success  can be also found in the treatment of attention deficit  disorder, pain  control and reduction in anxiety, which may occur during  long space  exploration.
 
\item[Who is doing it:] 
\hfill\begin{enumerate}
\item Omneuron demonstrated that   subjects were able to learn through training sessions to control pain   perception and pain control \cite{deCharms2008}
\item  Biofeedback training has   been adopted for over 20 years in training astronauts to resist motion  sickness \cite{Clement2008}.
\item Institute for Creative   Technologies of the University of Southern California uses VR to treat stress in Iraq war veterans \cite{Sutliff2005}. 
\end{enumerate}
 
\item[Time Scale:] 1 year
 
\item[Convergence:] The synergy   found where the creation of new computers, sensors, and communication   devices and research relating to new biological and medical  technologies  converge will extend the use of elaborate new tools to the  complex  operational environments facing today’s and tomorrow's  astronauts. 
 
\item[Significant Bottlenecks:]   Today's \gls{fMRI} machines offer the highest spatial resolution for the   noninvasive study of human brain activity. However, their large size  and  weight makes both inadequate for measurements in real-world  operational  environments. Currently, \gls{EEG} represents the only available  portable,  brain-imaging equipment suitable for this purpose; yet, it  has the  drawback of a relatively modest spatial resolution regarding  the source  localization of measured activity..
 
\end{description}

 
\subsubsection{Medical Treatments in Space:  Transcranial Magnetic Stimulation}
\label{neuro-treat}
\begin{description}  \item[Application:] Neurological Disease Treatment
 
\item[Problem:] Besides the late  effects of long term radiation exposure, stress, deprivation, isolation,  and confinement constitute major risks for the physical and mental  health of space explorers during the missions. However, limited  resources are available for medical treatments in space.
 
\item[Opportunity:]  \gls{TMS} is a noninvasive brain  stimulation technique which showed promising results in treating a range  of neurological diseases \cite{Ridding2007}.
 
\item[Exponential Technologies:]  Neural Technology and Computing
 
\item[Grand Challenges:] Staying There.
 
\item[Other  Connected Ideas:] Stem Cell Therapy
 
\item[Estimate of the Potential  Benefit:]Various conditions are currently treated successfully with  repetitive \gls{TMS}, such as stroke, Parkinson's disease, depression,  dystonia, tinnitus, epilepsy, amyotrophic lateral sclerosis,  schizophrenia, addiction, obsessive-compulsive disorder, Tourette's syndrome, and memory dysfunction \cite{Ridding2007}. \gls{TMS} is promising to provide  easier treatments to a variety of neural diseases in space given limited  medical resources.
 
\item[Who  is doing it:] \gls{TMS} is already \gls{FDA}-approved and widely used in depression  treatment \cite{Ridding2007}.
 
\item[Time Scale:] 1 year
 
\item[Convergence:]  In addition, utilizing the real-time noninvasive neural signals  recorded, neurofeedback and virtual reality will help the subjects to  learn how to control over their own brain activation  \cite{deCharms2008}. Success can be also found in the treatment of  attention deficit disorder, pain control and reduction in anxiety. 
 
\item[Significant  Bottlenecks:] \gls{TMS} stimulation spreads and lacks selectivity. 
 
\end{description}

 
\subsubsection{External Propulsion}
\label{extprop}
\begin{description}  \item[Application:] Launching payload into \gls{LEO} with external propulsion space launch  system
 
\item[Problem:]  Today space access is too expensive and unreliable as discuss in Space  Access Problem Space discussion.
 
\item[Opportunity:] Make space  access affordable for medium- and small-sized businesses; put space on  an exponential curve of growth through introduction of exponential  technologies into launch system
 
\item[Exponential Technologies:]  Material manufacturing and nanotechnology; computing; electronics
 
\item[Estimate  of the Potential Benefit:] Development and reduction to practice of  external propulsion space launch system can bring the cost of space  launch down to below \$1000 per kg into \gls{LEO}. Space launch can become  reliable, cheap and convenient. 
 
\item[Timescale:] 5-7 years
 
\item[Challenges:]  Development of a heat-exchange engine; development of beaming facility  that connects energy station on the ground with the launch vehicle;  public acceptance of a paradigm shifting technology.
 
\end{description}

\subsection{20-year Time Horizon}

\subsubsection{Space Based Solar Power}
\label{sbsp}
\begin{description}  \item[Application:] Collect solar energy from space and beam it to  Earth.
 
\item[Problem:] Today, with a  global power consumption of 12TW \cite{Seboldt2004}, we are reaching the end of the  energy supplies we have come to use so commonly. It is expected that by  year 2020 the global power consumption will be nearly 20TW. It is  becoming more expensive and more risky to search for fossil fuels to  power our cities, and the effects the combustion of these fuels is  having on the environment is becoming awfully damaging. Over 85% of the  power used today comes from fossil fuels \cite{Seboldt2004389}. It is  generally accepted that the time has come to focus on powering our world  on completely renewable resources, and there are many resources that  can be harnessed; tidal currents, wind, geothermal, and of course solar.  Solar power is unquestionably the ultimate solution to our energy  demands. In fact we have always relied on solar energy, we are just  prefer to use it in it's stored form of a battery called fossil fuels.


 
\item[Opportunity:] The most  promising way to harness solar energy to power the entire globe is  through a technology called Space Based Solar Power (SBSP). The idea is  not new, in fact it was first hypothesized nearly forty years ago. SBSP  simply takes advantage of the large amount of solar energy striking the  Earth from orbit, which is 1.366 kW/m$^2$. A SBSP plant will collect  gigawatts of energy in orbit, electromagnetically beam it to the Earth  at a suggested frequency of 2.45 GHz \cite{Chaudhary2010}, and then use it in a multitude  of forms. A single kilometer wide band of geosynchronous Earth orbit  receives enough solar flux annually to equal that of all the energy  within all the known oil reserves on Earth today \cite{NationalSpaceSociety2007}.
 
\item[Estimate of the Potential  Benefit:] SBSP can provide solar energy to anywhere on Earth 24 hours a  day. Every nation on Earth will be taken out of the dark ages once this  system becomes viable.
 
\item[Who is doing it:] Space  Energy is a commercial company building a prototype SBSP system to be  launched to LEO  \cite{SpaceEnergy2010}. JAXA also has plans to build a SBSP  program.
 
\item[Current Bottlenecks:] There  are several advancements that Exponential technologies could deliver  that would enable SBSP to be economical, at which point it will  unarguably be undertaken. The first obstacle is access to space. If a  new form, or at least cheaper form, of space access were to be  implemented the materials to build the SBS plant could be taken to  space.
 
Another option however would be to  utilize the materials on the moon, asteroids, or space debris to  construct the SBS plant. Since the current launch cost is 10,000 dollars  per Kg, as soon as materials are brought from anywhere but Earth the  cost of the endeavor decreases by nearly 10,000 dollars! In fact, a  Lunar Power Beaming Station could be most effective as lunar materials  would be used to build the station without having to move the materials  off the lunar surface, however the timescale on such a project is far  off. This is because with todays technology it would be too costly to  build the structures on the moon, transmit the power to GEO satellites,  that could then redirect the power to rectennas on Earth. It also seems  unlikely that a government space program will focus on this goal in the  near term, and it would take a government agency to afford such a  project \cite{Seboldt2004}.
 
Another obstacle limiting SBSP is  that most plans suggest photovoltaics as the source of capturing solar  energy to then be converted to microwave. \gls{PV} technology is however quite  inefficient, with maximum efficiencies on space rated \gls{PV} cells at  around 25%. Sure, the efficiencies of \gls{PV} is increasing continuously, but  there is a limit. Fortunately there are other ways of harnessing solar  energy, that can also reduce the necessary size, weight, and cost of the  SBS plant. Concentrating solar energy to a large Stirling engine (see \autoref{stirling}) is one  example. This concentrated energy will provide even higher temperature  extremes than normal and would thus create an abundance of energy. The  only downside is that while \gls{PV} converts solar energy directly to  electricity, the sterling engine however creates mechanical energy that  must then be converted to electrical energy with losses. Still though,  concentrated solaris currently more efficient than \gls{PV}'s.
 
\end{description}


\subsubsection{Space Based Stirling Engine}
\label{stirling}
\begin{description}   \item[Application:] A novel method to harness solar energy on the moon,   asteroid, or space using sterling engines
 
\item[Problem:] Are there other   ways to harness the power of the sun other than photo voltaic? A method   that is more efficient and possibly offers a better means of energy   storage?
 
\item[Opportunity:] Stirling   engines offer an efficient means of energy production but are limited  in  use due to the large temperature differential needed, the  environment  of space however offers the temperature differentials  needed to make  this form of energy production applicable. 
 
For Stirling engines, the energy   can roughly be estimated as $E \approx C_{p}(T_{2}-T_{1})$, where $E$ = energy, $C_p$ =  pressure  coefficient, and $T_2 - T_1$ = temperature differential. Since no   combustion takes place within a Sterling engine the energy output is   much less compared to the combustion engine for most conditions,   however, when the temperature differential is on the order of hundreds   of degrees Kelvin we find larger energy outputs. 
 
On the lunar surface the   temperature difference between the shaded and unshaded regions is   roughly 530 K. A Stirling engine could operate on the  lunar surface with  the hot side facing the direction of the sun and the  cold side being on  the underside in the shadow of the device. With the  use of energy  storage devices a lunar Sterling engine could supply  energy to a lunar  base while also storing residual energy to be used  during the Lunar  night. A lunar day/night lasts 14 Earth days.
 
The same idea can be applied in   open space or orbit with a bit of attitude control to keep the hot and   cold sides in proper orientation. And with advancements in  nanotechnology, such as  Buckypapers, the hot and cold side of the device  can be constructed to  be perfectly absorptive or reflective to maximize  temperature extremes.
 
\item[Estimate of the Potential   Benefit:] When a lunar base is established it will be very important to   harness as much solar energy as possible during the 14 Earth day long   Lunar day since the Lunar night is just as long. Storing energy for the   lunar night is also important. The sterling engine offers some unique   methods to store the extra energy for the lunar night and possibly   create more abundance of energy compared to \gls{PV}'s.
\end{description}
 

   \subsubsection{3D Printer in Orbit}
  \label{3dprint}
 \begin{description}  \item[Application:] a \gls{3D printer} in orbit will be able to print antennas,
reflectors  and solar cells.
 
   \item[Problem:] Sending up antennas, reflectors and solar cells to orbit  is
both expensive, difficult and  inefficient. Expensive in proportion to
the  weight, difficult to build structures that will stand the rigour of
launch,  and inefficient because the structures are built to support high
accelerations  and high Gs, while once in orbit all that structural mass
is  unnecessary.
 
   \item[Opportunity:] A \gls{3D printer} in orbit could print lighter antennas,
reflectors  and solar cells without cumbersome structural restrictions,
using  minimum materials. The supply materials to the \gls{3D printer} could be
either  sent from Earth (in high gravity launches); refurbished from existing
orbiting  spacecraft, antennas and reflectors; or eventually provisioned
from  asteroid mining.
 
   \item[Exponential Technologies:] Nanotechnology, Robotics.
 
   \item[Grand Challenges:] Robotic Exploration, Human Exploration,  Staying There
 
  \item[Other Connected Ideas:]  Asteroid Mining; 3D Printing; In-Orbit
Refurbishing
 
  \item[Estimate of the Potential  Benefit:]Lighter and bigger structures in
space for communications, energy  harvesting and others.
 
   \item[Who is doing it:] -
 
   \item[Time Scale:] 10-15 years.
 
   \item[Convergence:] Robotics, Computing and Networks, Materials, AI.
 
   \item[Significant Bottlenecks:] Making a 3D printer work in  $\mu$gravity, robotic manipulators to extract and assemble 3D-printed  parts without humans present
     \end{description}
     
     \subsubsection{3D  Printing in Space} 
\begin{description}  \item[Application:] 
3D  printing space exploration tools and parts on an ``as-needed'' basis.
 
\item[Problem:] 
Manufacturing space exploration  tools is a long, laborious process. A simple example is in which a basic  tool bag can take months to years to build and cost over \$100,000 \cite{Bryner2008}.
 
\item[Opportunities:] 
Some metals, ceramics, and  plastics can be printed with current 3D printing technologies here on  Earth, and the palette of materials is expanding exponentially. However,  the current 3D printing model will not work in microgravity  environments. The technologies today use an electron beam in a vacuum  chamber to locally melt parts of a layer of metal dust to build up the  structure voxel by voxel, layer by layer. Without gravity to form the  layer of metal dust, this technique is impossible. 
 
\item[Who is doing it:] 
A new technique called Electron  Beam Freeform Fabrication (EBF3) pioneered at NASA Langley overcomes  this challenge by extruding a metal directly from a tip \cite{Dillow2009,TechnologyGateway2008,Banke2009}. This is a promising technology but needs to improve:  the extruding machinery needs to become lighter and its resolution needs  to increase. However these are merely problems of engineering that will  improve with the exponential increase in the strength of materials. 
 
\item[Enabling technologies:] 
Exponential technologies that will  impact 3D printing in space include: 
\hfill\begin{enumerate}
\item Network and computing  systems (leading to faster fabrication). 
\item AI \& robotics  (automating the process of creating the software on a tool that needs to  be created. For example, a 3D scan a machine that is being repaired,  and then layout the mock-up of the tool needed to repair the machine). 
\item Nanotechnology (creating a  more precise fabrication process and less waste of materials). 
\end{enumerate}
 
\item[Time Scale:] The Electron  Beam Freeform Fabrication (EBF3) equipment has been successfully tested  on the ground \cite{Taminger2006}. Although the ground equipment is fairly bulky, a  scaled down version of EBF3 was created and flown to in  near-microgravity on NASA's Reduced Gravity Aircraft \cite{Dillow2009}.
 
\item[Example 3D Printing  Scenario:]
In the near term, the greatest use  of 3D printing in human space missions is printing tools or components  as they are needed. Custom-built tools means fewer tools taken to orbit  with less redundancy in number of tools taken. In the longer term, 3D  printers are compelling and enabling tools for human colonization. The  ability of either robots or humans to be able to print what is necessary  using local materials is a compelling reason to pursue this technology.

   \end{description}
 
 \subsubsection{3D Printing Cheap Small Mining Craft}

\begin{description}

\item[Application:]

Generating small, cheap, fairly reliable mining spacecraft for obtaining
valuable and essential resources from asteroids.


\item[Problem:]

Current manufacturing processes are expensive and require a great
deal of manpower. Therefore, extreme care must be taken to make sure
that asteroid mining missions do not fail, which drives the cost up
even more. If mining spacecraft could be produced at a fraction of
the cost (and a fraction of the size), one mission (one payload) could
easily launch 100 or even 1,000 mining spacecraft, each to a different
asteroid. This would dramatically increase the chance of success for
the mission overall, as well as the fruits of each mission, while
simultaneously decreasing the cost of the mission as a whole.


\item[Opportunity:]

Three-dimensional printing technology has already dramatically cut
the costs and time involved in manufacturing products in several industries.
As the field advances, 3D printers will be capable of printing with
more materials, cheaper, and faster. Shortly, these printers will
be capable of simply printing entire mining spacecraft. Effectively,
the impact will be equivalent or even better than instantaneously
erecting a mass-manufacturing plant for mining spacecraft for \$0.
As a result, the cost of producing small and effective mining spacecraft
will drop sharply.


\item[Exponential Technologies:]

Nanotechnology, Computing


\item[Grand Challenges:]

Robotic Exploration, Staying There, Resources


\item[Other Connected Ideas:]

In order to make this idea work, it will also be necessary to combine
the 3D printing technologies with miniaturization, sufficient automation,
and robotics.


\item[Estimate of the Potential Benefit:]

It is estimated that there are trillions of dollars worth of materials
available in near-Earth asteroids. Bringing these materials back successfully
would not only greatly benefit the mining entity financially, but
it would also unlock billions of people on the planet from being restricted
by the terrestrial shortage of the materials that are abundant in
asteroids.


\item[Who is Doing It:]

\hfill\begin{enumerate}
\item Professor Hod Lipson at the Cornell Computational Synthesis Library
heads the Fab@Home project, which is an open source project that has
already resulted in the 3D printing of part of a working robot in
2009\cite{fabathome}.
\item Hod Lipson, along with Jordan B. Pollack, actually succesfully created
robots using 3D printing technology as a part of The Golem Project
at Brandeis University in the year 2000\cite{golem}.
\item Reshko, Mason, and Nourbakhsh have successfully used rapid-protoyping
technology to create small robots in the year 2000\cite{reshko}.
\item Won, DeLaruentis, and Marvodis at Rutgers University are also working
on rapidly prototyping robotic systems\cite{won}.
\end{enumerate}

\item[Time Scale:]

We are just now seeing rapid prototyping technology over the last
few years that is capable of producing components of mining machines
already. We project that within the next 20 years it will be possible
to rapidly prototype an entire robotic mining system using 3D printing
technology..


\item[Convergence:]

This opportunity lies at the intersection of advances in 3D printing,
design software, materials, microprocessor miniaturization, and robotics.


\item[Significant Bottlenecks:]

Rapid prototyping and 3D printing is accelerating a fast clip. Since
the 3D printing of parts of a spacecraft is already highly feasible,
there are no significant bottlenecks that would have to be overcome
to achieve this goal.
\end{description}
 
 \subsubsection{AI-powered Mining Craft}
 \label{auto-explore}
\begin{description}


\item[Application:]

Mining asteroids cheaply by letting the mining spacecraft handle most,
if not all of the decision-making.


\item[Problem:]

Presently, all space missions require a certain degree of human control.
By cutting humans out of the loop, asteroid mining becomes much more
viable, not only because it is cheaper, but because thousands of missions
can be in operation concurrently.


\item[Opportunity:]

By leveraging contemporary advances in AI and robotics, it will be
possible to create cheaper asteroid mining systems that perform better
on their own, without human intervention, than perhaps even human-guided
missions would. Since asteroid mining consists primarily of performing
mechanical tasks, contemporary AI and robotics greatly advance this
field.


\item[Exponential Technologies:]

Computing, AI \& Robotics


\item[Grand Challenges:]

Robotic Exploration, Staying There, Resources


\item[Other Connected Ideas:]

This project is connected closely to the project of 3D-printing and
rapidly prototyping asteroid mining spacecraft.


\item[Estimate of the Potential Benefit:]

The ability to launch a swarm of concurrent, cheap asteroid mining
missions without human intervention will allow the asteroid mining
problem to finally become economically viable.


\item[Who is Doing It:]
\hfill
\begin{enumerate}
\item Countless academic research groups and companies are working on producing
better autonomous robotic systems. For a good overview, see Bekey's
\emph{Autonomous Robots}, published in 2005, which surveys over 300
contemporary systems\cite{bekey}.
\item As an example for a future-looking autonomous robotics company, Willow
Garage, a company based in Menlo Park, CA, is working on developing
a standardized robotic operating system as well as a general-purpose
robot that could be eventaully applied for usage in asteroid mining.
They have just this year given 11 unites to leading research institutions
in robotics\cite{willow}.
\end{enumerate}

\item[Time Scale:]

There are already many narrowly-intelligent artificial systems, as
well as many successful robotic applications here on Earth and in
space. It is estimated that in less than a decade, these technologies
will be applied commercially to asteroid mining.


\item[Convergence:]

This opportunity lies at the convergence between Artificial Intelligence
and robotics.


\item[Significant Bottlenecks:]

While there are many significant bottlenecks inherent in designing
better AI and better robotics, the AI and robotics systems that are
currently present on the Earth in 2010 are more than sufficient to
dramatically advance the asteroid mining endeavour.
\end{description}
 
 \subsubsection{Asteroid Composition Detection with Microbes and Crystallography}
\begin{description}
\item[Application:] Engineered fluorescent microbes and \gls{crystallography} to detect asteroid composition for asteroid mining
\item[Problem:] Mining an asteroid entails a huge investment of money, efforts and time. Also, the composition of an asteroid varies much and selecting the right one to mine is a very important step. Heavy metals, some of which very rarely occuring on Earth, i. e. platinum, are heavily represented in certain types of asteroids \cite{Kargel} which should therefore be preferentially picked for mining purposes and the contents brought back to Earth. Importantly, the current periodic table of the elements could be revisited and a new one might be created with all the new elements, if any, that asteroids might carry
\item[Opportunity:] Exploit the raw materials from asteroids to address the current shortage of supply of heavy metals on Earth and discover new elements that have not to date reached the Earth's surface that might be represented on the surface of asteroids. A way to detect the composition of the asteroid before docking, is to release engineered microbes that, if in contact with specifically sought for metals/materials, would fluoresce at a particular wavelength corresponding to a specific color. If multiple microbes servicing to detect specific materials were to be released at the same time, then the asteroid would start fluorescing at a combination of wavelengths that could be detected by a \gls{CCD} camera and resolved in the single components. The intensity of each fluorescence, given the same amount of microbes is released, will indicate the percent composition for each material within the asteroid. Detection of new materials could be addressed by having a \gls{crystallography} machine on board that could produce an image of all the crystals (i. e. solid array of atoms) on the asteroid.

\item[Exponential Technologies:] Synthetic Biology, \Gls{crystallography} (X ray, powder diffraction, fiber diffraction, small-angle), \gls{TEM}, Communication technology (i. e. space internet)
\item[Grand Challenges:] Human Exploration
\item[Other Connected Ideas:] The delivery of the microbes and the minining could be done through machines with robotic arms along the same lines of the ``Da Vinci'' ones developed by Intuitive Surgical, which could be remotely controlled from Earth
\item[Estimate of the Potential Benefit:] Filling in on the shortage of resources on Earth and discover new and perhaps useful elements to bring back to Earth for specific purposes, i.e. metallic and materials industry
\item[Who is doing it:]
\hfill\begin{enumerate}
\item NASA stardust mission
\item Japan's Hayabusa probe
\item European Space Agency's comet-chasing Rosetta spacecraft mission
\end{enumerate}
\item[Time Scale:] 10 years
\item[Convergence:] AI, Robotics, Communication technology
\item[Significant Bottlenecks:] Docking the asteroid to start the mining, communication lag from asteroid to Earth if robotic arms were to be remotely controlled, synthetising microbes that have sensitivity for the right metals to be analysed
 \end{description}
 
    \subsubsection{Lunar Habitat-Building Robot}
 \begin{description}  \item[Application:] A lunar-based robot that can build structures with
vitrified regolith for  lunar-habitat development
 
  \item[Problem:] Preparing  habitats for human colonies in the moon requires
structural materials to be either  shipped to the moon or fabricated
locally.
 
  \item[Opportunity:] Mixing small  metallic nano-particles with lunar
regolith  and applying microwaves to heat and vitrify the mix, or other
means of regolith vitrification  could be used to produce
macro-structures  to be integrated into habitat design.
 
  \item[Exponential Technologies:]  Robotics, Nanotechnology
 
  \item[Grand Challenges:] Staying  There, Human Exploration, Robotic Exploration
 
  \item[Other Connected Ideas:] 3D printing in space
 
%  \item[Who is doing it:] -
 
  \item[Time Scale:] 10--20 years.
 
  \item[Significant Bottlenecks:] Technology development
   \end{description}
  
\subsubsection{Multi-Surface Robots}
\begin{description}  \item[Application:] New range of various robotic designs that are able  to transverse lands or oceans.
\item[Problem:]  few rovers and robots have been sent for space exploration. Many  extrasolar planets may be covered with water, in the future ``swimmer  robots'' would be of extreme use.
\item[Opportunity:] 
\item[Grand Challenges:] Robotic  exploration, staying there
\item[Other Connected Ideas:] swarm intelligence
\item[Estimate of the Potential  Benefit:] 
\item[Who is doing it:] Romela lab (Virginia Tech) is  working on new design ideas \cite{AUV2010}
\item[Time Scale:] 5--15 years.
\item[Convergence:] 
\item[Significant Bottlenecks:]  Design 
\end{description}
 
\subsubsection{Brain Machine Interface for Robotics}
\label{bmi}
\begin{description}  \item[Application:] Multitasking at work and communication
 
\item[Problem:] Limited labor in  starting up a new colony in space.
 
\item[Opportunity:] Brain machine interface should be utilized to remote control multiple robots and  communicate directly with the crew. Researchers have already  demonstrated that we can control computer cursors and even robotic arms by decoding neural signals, such as the DEKA arm. Brain waves were also  demonstrated to be able to be able to classify a subset of words in  internal speech \cite{Suppes1997}.
 
\item[Exponential Technologies:]  Neurotechnology, Biotechnology, Computing
 
\item[Grand Challenges:] Human Exploration, Staying There
 
\item[Other Connected Ideas:]  Robotics and Communication
 
\item[Estimate of the Potential  Benefit:]Real-time decoding of the resulting neural signals will help us  to handle multiple tasks and communicate with multiple colleagues more  efficiently.
 
\item[Time Scale:] 10-20 years.
 
\item[Significant Bottlenecks:]  Currently, EEG represents the only available portable, non-invasive  brain-imaging equipment suitable for this purpose; yet, it has the  drawback of a relatively modest spatial resolution regarding the source  localization of measured activity.
 
\end{description}

\subsubsection{AI and Synthetic Biology}
\label{aisyn}
\begin{description}
\item[Problem:]
Sustaining life on other planets  would require establishing basic living conditions, including gravity  and radiation protection, providing basic human needs, including  breathable air, water and food, and maintaining a clean healthy living  environment. This also partially 
applies  to long space trips. Synthetic Biology enables designing new organisms  that could help in many of the areas above. However, there is a gap  between the DNA codes we can ``write,'' and the real world requirements.
 
 \item[Opportunity:] AI could help bridge that gap.  Treating DNA as a language, with its  syntax, grammar and semantics, AI could enable definition of organism  ``functionality'' in ``natural language'' - high level description, which is  automatically translated all the way to DNA that can be ``written'' into a  new organism. Such an AI driven Synthetic Bio system would take into  account all the constraints, at all levels - from gene's, through RNA,  DNA, and folding, all the way to the
 phenotype  level. 
 
 \item[Exponential  Technologies:]  Synthetic Biology, AI bringing the paradigm shift
 
 \item[Grand Challenges:] Staying There, Getting There
 
 \item[Estimate of the Potential  Benefit:] Synthetic organisms could help  producing and recycling air, water and food, as well as facilitate  terraforming.

 \item[Time scale:] Currently we know how to write  around 1 million genes, 
sufficient  to modify single cell organisms. As an exponential technology, we  expect to be able to design single-cell organisms within 5 to 10 years  simple multi-cell organisms within 10 to 15 years. 
 
\item[Convergence:] Synthetic Bio could eventually  lead to enhanced humans, and human-like bio-robots.

\item[Significant Bottlenecks:]
\hfill\begin{enumerate}
\item Much of the constraint and mapping  space between DNA level and high
 level  functions is not known.
 \item Resistance to artificial life  forms and fear of the perils it could involve.
 \end{enumerate}
 \end{description}
 
\subsubsection{AI driven Synthetic Biology for space exploration} 
\begin{description}
\item[Problem:]
Space exploration would require a  human friendly environment maintained inside a spaceship, or in a small  base on another planet, for an extended period of time. This includes  provision of clean air, clean water, and food.
 
\item[Opportunity:]
Synthetic Biology enables the design and  manufacturing of new organisms that 
would  help:
\hfill\begin{enumerate}
\item creating and recycling air:  synthetic organism/bacteria that use toxic atmosphere to produce oxygen,
\item purifying sewer water: synthetic  organisms that feed on the contaminating materials
\item producing food: synthetic  organisms that process nutrient to produce food in solid or liquid form  (such as milk, pills with specific properties)
\item In the context of space  exploration, typically involving only few people, such solutions can be  modular and movable, containable small/ confined/ constrained spaces -  tanks, rooms, etc...
\end{enumerate}
 
\item[Exponential  Technologies:]  AI, Synthetic Biology
\item[Grand  Challenges:] Staying  There, Getting There.
\item  [Estimate of the Potential Benefit:] Consistent, cheap and light supply  of human life needs.
\item[Time  Scale:] 
Currently we know how to write  around 1 million genes, sufficient to modify single cell organisms. As  an exponential technology, we expect to be able to design single-cell  organisms within 5-10 years simple multi-cell organisms within 10-15  years. 
\item[Convergence:]
Synthetic biology could eventually  lead to enhanced humans, and human-like bio-robots.
\item[Significant  Bottlenecks:]
\hfill\begin{enumerate}
\item There is a gap between the  DNA codes we can ``write'' and the real world requirements. AI could help  bridge that gap.
\item Resistance to artificial  life forms and fear of the perils it could involve.
\end{enumerate}
\end{description}
 
\subsubsection{AI and Synthetic Biology for Colonizing Other Planets}
\label{terraform}
  \begin{description}
\item[Problem:]Colonizing other planets would  require establishing proper large scale living conditions for an  increasingly large population, including:
\hfill\begin{enumerate}
\item Supply of air, water and food.
\item Radiation protection
\end{enumerate}
 
This would require massive change of  the environment to suit human existential needs - terraforming. 
 
\item[Opportunity:]Synthetically created organisms  could help terraforming other planets 
by  including:
\hfill\begin{enumerate}
\item Generation of an atmosphere by  transforming available materials into air.
\item Transforming local soil 
\item Synthetic plants designed to grow  in the new environment, making a food source.
\item Generate water from locally  available materials. 
\end{enumerate}
 
 \item[Exponential  Technologies:] AI,  Biotechnology
  \item[Grand Challenges:] Staying There, Getting There
  \item[Estimate of the Potential  Benefit:]Providing human life necessities by transforming locally  available materials, requiring a load of bits (information) rather than  atoms (matter).
  \item[Who is doing it:]
 \item[Time Scale:] Currently we know how to write around 1 million genes,  sufficient to modify single cell organisms. As an exponential  technology, we expect to be able to design single-cell organisms within  5-10 years simple multi-cell organisms within 10-15 years. 
  \item[Convergence:] Synthetic Bio could eventually  lead to enhanced humans, and human-like bio-robots.
  \item[Significant  Bottlenecks:]
  \hfill\begin{enumerate}
  \item There is a gap between the DNA  codes we can ``write'' and the real world requirements. AI could help  bridge that gap.
  \item Resistance to artificial life  forms and fear of the perils it could involve.
  \end{enumerate}
\end{description}

\subsubsection{Bioreactors for Terraforming}
\label{bioreactors}
 
\begin{description}   \item[Application:] The use of microorganisms genetically modified  could
improve the life support systems   which are necessary in the process of
soil   terraforming on other planets in order to reach a sustainable
production  of food, water, oxygen,  and waste treatment.
 
\item[Problem:]  Last year marked  the 40th anniversary of our first steps on
the  Moon, and within next decades  it is hoped that humankind will have
established   a settlement on Mars. Space is a harsh environment, and
technological  advancements and  operational challenges, such as the
management   of life-support systems, food, waste disposal, and
nutrition  due to long-term  confinement will be required to ensure that
humans  survive interplanetary  journeys and settlements \cite{Horneck2010}. Because of   this, one main challenge will be to redefine environments that cannot at the   moment host life, as we know it, and turn them into habitable ones, so   as to ensure the first human
colonization   into space   \cite{gerlach}.
 
Due  the life support systems do  not permit long staying into space,
and  the  models of terraformation could take ~100 years or more, new
ideas  and options using the  exponential technologies are needed in
order   to improve the process and shorten the time \cite{Heppener2008}.
 
\item[Opportunity:]  A tiny  fraction of Earth’s organisms have been exposed to
the  harsh conditions of  spaceflight. Many specimens that have
experienced   spaceflight were noticeably changed by the event. Space
biology  has emerged as a critical  component of successful human space
exploration,   fundamental biology research, and our understanding of
the  limits of life \cite{Dubertret1987,Hendrickx2007}
 
Although,  there are several  analogs of environments and models of
possible   loop close systems \cite{Poughon2009}, the use of
microorganisms  genetically  modified with specific functions could
improve   the efficiency and performance of the system \cite{Kern2001}.
 
\item[Exponential  Technologies:]  Biotechnology (DNA sequencing, genetic manipulation).
 
\item[Grand  Challenges:] Staying  There
 
\item[Other  Connected Ideas:]  Genetic manipulation in plant, mammalian, and
human  cells for similar systems.  Synthetic biology.
 
\item[Estimate  of the Potential  Benefit:] The possibility of becoming any
extreme  or harsh environment which  is not suitable for human life into
a   habitable one.
 
Genetically  engineered  microorganisms have higher performance and
better  results in specific  functions compared to their corresponding
wild  strains \cite{Stellwag1986}.
 
\item[Who  is doing it:]
\renewcommand{\labelenumi}{\arabic{enumi}.   }
\renewcommand{\labelenumii}{(\alph{enumii})}
\hfill\begin{enumerate} 
\item Genetic manipulation of   microorganisms:
\begin{enumerate}
\item Group of Dr. Donald A. Bryant,   Principal Investigator and Professor
of   Biotechnology, Biochemistry and Molecular Biology. Pennsylvania
State  University. USA. Since 2006,  he have carried out pioneering
research   on the genetic manipulation of four phyla of phototrophic
bacteria:  Cyanobacteria, Chlorobi  (green sulfur bacteria), Chloroflexi
(filamentous   anoxygenic phototrophs), and the newly discovered
Acidobacteria.
 
\item Group of Dr. David Dubnau,   Investigator in Public Health Research
Institute   Center, UMDNJ - New Jersey Medical School. Since 1994, he is
working  on the mechanisms used for  bacteria to take up environmental
DNA  in  macromolecular form in a process known as transforming. He uses
genetic  engineering in order to  improve different functions in
bacteria   as well.
 \end{enumerate}
\item Models of Terraforming and   bioreactors:
 \begin{enumerate}
\item  European Space  Agency and Belgian  Nuclear Research Centre (SCK CEN)
projects   (2010).
 \begin{itemize}
\item  The BASE (Bacterial Adaptation   to Space flight Environment)
project   aims to characterize the behavior of bacteria under
spaceflight  conditions including  cosmic radiation and microgravity.
 
\item  MELGEN-2 (Melissa Genetic   Stability study) is involved in the
development   of novel methods to detect metabolic/genomic instability,
microbial  contaminants and  horizontal gene transfer in the MELiSSA
loop.  MELiSSA (Micro-Ecological  Life Support System Alternative) is a
multidisciplinary   project of the European Space Agency ESA. It aims at
the  development of a  bioregenerative life-support system to enable
future  long duration manned space  missions (e.g. to Mars) by
reconversion   of organic gas, liquid and solid wastes into oxygen,
water  and food. Proper functioning  of the MELiSSA loop will be
dependent   on the stability and axenicity of each of its compartments.
 
\item  MISSEX (Microbial Gene  exchange  in the International Space
Station)   project has as aim to study the micro-organisms that are
present  in confined space ships or  space stations.
\end{itemize}

\item Group of Prof. Jeffrey T. Richards, Dynamac Corporation, University
of  Florida, Space Life Sciences  Laboratory, Kennedy Space Center,
Florida.   He is working in hypobaric environments and its relationship
with  microorganisms evaluating the  implications for low-pressure in
life   support systems during human exploration missions and
terraforming  on Mars \cite{Richards2006}.
\end{enumerate}
\end{enumerate}
\renewcommand{\labelenumi}{(\alph{enumi}}
\item[Time  Scale:] The genetic  manipulation is been done in the laboratory
for  several years  \cite{Stellwag1986}.  However, the design
of bioreactors for life support   systems and terraforming, is still
theoretical   or has different uses on Earth. The real application could
be  in the near-term (~5 years)  using the new microorganisms.
 
\item[Convergence:]  Robotics to  ensure automatic manipulation of life support
system  bioreactors. Nanoparticles  could mimic the bacteria functions.
Synthetic   biology.
 
\item[Significant  Bottlenecks:]  Several models of terraforming, principally
Mars,  using microorganisms assume  that the initial stage of planetary
engineering   has been accomplished, and has a denser atmosphere, in
which  liquid water is stable, and a  higher average surface temperature
such  as  the case of Mars \cite{McKay2001}.
In  addition, the use of  microorganisms as a trigger of terraformation
present  several ethical issues  associated with bringing life to other
planet   center on the possibility of indigenous life and the relative
value  of a planet with or without a  global biosphere \cite{Debus2008}.
\end{description}
 

\subsection{30-year+ Time Horizon}

\subsubsection{Nanotech  Space Suit } 
\label{nano-suit}
\begin{description}  \item[Application:] A lightweight, mobile spacesuit of tomorrow
 
\item[Problem:] The spacesuits  currently in use weigh hundreds of Earth pounds, take up too much space,  and are extremely immobile. The immobility of the space suit and it's  large mass is due to the fact that it has to protect the astronaut form  the harsh environment of space. 
 
\item[Opportunity:] Protecting  from radiation is extremely important. Carbon Nanotube membranes could  be used to make a very thin material to be worn as a radiation garment,  similar to a wetsuit, but not necessarily skin tight. 
 
\item[Estimate of the Potential  Benefit:]Will allow future space explorers more freedom while doing  extra vehicular activities.
\end{description}

 
     \subsubsection{Nanoengineered Smart Materials}
 \begin{description}  \item[Application:] Smart materials for space habitats and spacecraft  with integrated sensors, morphing and self-healing properties.
 
  \item[Problem:] Space habitats  and spacecraft are structures built in the
limit of technical capability, and  subject to multiple hazards.
 
  \item[Opportunity:] Using  nano-scale engineering, smart materials can be
developed that include, in the  same structure, sensors for structural
stability  and habitat conditions, and that can adapt rapidly at
different conditions.  Nanomaterials with the ability to self-heal, morph
for different functionalities and  respond in real time to threats like
radiation,  or micrometeorites.
 
  \item[Exponential Technologies:]  Nanotechnology, AI.
 
  \item[Grand Challenges:] Staying  There, Human Exploration, Getting There
 
  \item[Other Connected Ideas:]  Nano-Fabricators
 
  \item[Estimate of the Potential  Benefit:]Lighter, safer structures for use
in space, or getting there.
 
  \item[Who is doing it:] -
 
  \item[Time Scale:] 15--25 years.
 
  \item[Convergence:] Nano,  Computing and Networks, AI.
 
  \item[Significant Bottlenecks:] Nanoassembly technology.
    \end{description}
 
 
\subsubsection{Cheap Access / Space Elevator / Carbon nanotubes}
\begin{description}  \item[Application:] Cheap Reliable Access To Space 
 
\item[Problem:] The cost of space  access is too high for major developments to take place 
 
\item[Opportunity:]  With advancements in carbon nanotubes a geosynchronous satellite can be  used as a space elevator, lifting payloads to orbit on a carbon  nanotube wire.
 
\item[Estimate  of the Potential Benefit:] Opens the space frontier
 
\item[Bottlenecks:] Carbon  nanotubes that are kilometers long need to be invented before this  technology has any chance of happening.

\end{description}
 
 
    \subsubsection{Brain Machine Interfaces for Remote  Sensing/Presence} 
\begin{description}  \item[Application:] -
  
   \item[Problem:] The problem with today's external \glspl{BMI} is their limited  resolution, inability to 
connect well to the inner parts  of the brain, usability problems due to an abundance of sensors 
to  be attached. Neurally connected \glspl{BMI} afford surgery operations. \Gls{fMRI} and other imaging 
technologies  provide better resolutions, increasingly in real-time fashion.
  
  \item[Opportunity:] Portable as  well as implanted \glspl{BMI} with neural resolution can provide a much 
richer remote sensing experience  compared to today’s remote presence technologies (HMD, 
Caves, Immersion technologies,  mixed reality). If remote robotics provide rich sensing 
opportunities from exploration,  such experience can be reflected within the \gls{BMI} in a much 
better fashion than today.
 
    \item[Exponential Technologies:]  Nanotechnology, Biotechnology,  Neurotechnology, Computing.
 
  \item[Grand Challenges:] Robotic  Exploration 
 
   \item[Other Connected Ideas:] Neurobiology, Nano/Materials Science.
 
 \item[Estimate  of the Potential Benefit:] Full immersion with regards to remote  places, wider 
bandwidth of sensing experience  through real-time filters, real-time augmentation of experience.
 
   \item[Who is doing it:]
  \hfill\begin{enumerate}
    \item Fraunhofer IGD
     \item MIT MEDIA LAB
     \item Merl
     \end{enumerate}
 
   \item[Time Scale:] 5 years limited visual/retinal,10 years partial  real-time, 30 years full immersion
 
   \item[Convergence:]  Nanotechnology, Neurotechnology
 
   \item[Significant Bottlenecks:]
Complete mapping of the brain,  lasting sustainable neural/electronic connection. Representation 
resolution. High bandwidth  connection to remote areas in space.
    \end{description}
 
 
 
\subsubsection{Cheap  swarm  spacecraft for exploration}
 
 \begin{description}   \item[Application:] Mediated human exploration of space through the
extension of our bodies into   swarms of robots and spacecraft.
 
  \item[Problem:] Exploring the   solar system for fun and profit (both for
research  and prospecting  purposes), will be regarded more and more as a
fundamental activity in the  coming  decades. Findings will lead to great
riches,   and people will invest heavily in this new gold-rush. However,
there are limits to human   exploration: it is more risky, more expensive,
and  humans can only be in one  place at a time
 
  \item[Opportunity:] Building   sensory interfaces into our spacecraft and
robots  that can extend an  operator's sensory input to include feedback
from a remote swarm of robotic   explorers and unmanned spacecraft can
potentially   extend the reach of our bodies well into the solar system,
allowing for a mediated, yet  human  exploration of space through other means.
 
  \item[Exponential  Technologies:]  Neurotechnology, AI, Biotechnology,
Robotics,   Networking.
 
   \item[Grand Challenges:] Robotic  Exploration, Human Exploration,  Getting
There, Staying There, Using Space   Resources.
 
   \item[Other Connected Ideas:]  Asteroid Mining; Robotic Exploration  Swarms;
 
%  \item[Estimate of the  Potential  Benefit:] 
 
   \item[Who is doing it:] -
 
   \item[Time Scale:] 30 years--
 
  \item[Convergence:] Bits, Neuroscience and Genetics, Robotics and AI
 
  \item[Significant Bottlenecks:] Latency-tolerant brain-machine   interfaces and networks.
    \end{description}

 
\subsubsection{Emotionally  intelligent robot/humanoid for astronauts}
\label{companion}
\begin{description}  \item[Application:] Developing an Emotionally Intelligent  Robot/Humanoid that serves as an astronaut companion to mitigate the  effects of prolonged staying in confined spaces.
\item[Problem:]  It is hard to stay in confined space for an extended amount of time  with other people. Although astronauts are required to be socially  outgoing persons who are excellent team players, and are cable of  adapting to new changing environments, stress induced from such a  compact hight risk environment is a probable outcome, and can affect the  astronauts' self control and judgment.
\item[Opportunity:]  New researches in emotional intelligence and how to program the robots  to sense physical gestures and facial expressions, perceive emotions,  and interact with other human beings lead the way to a new definition of  human-robot interaction that has its results in ameliorating difficult 
\item[Exponential  Technologies:] Computer Networks, Narrow AI
\item[Grand  Challenges:] Getting There
\item[Other Connected Ideas:]  Artificial General Intelligence, materials, sensors
\item[Estimate  of the Potential Benefit:] Reduced levels of stress help astronauts in  their mission
\item[Who is doing it:]
\hfill\begin{enumerate}
\item Scientists at UC San Diego's  California Institute for Telecommunications and Information Technology  (Calit2)-Hanson Robotics of Dallas, Texas
\item Carnegie Mellon University's Robotics  Institute, Human Computer Interaction Institute, and the Entertainment  Technology Center.
\item MIT Media Lab-Personal Robots
\end{enumerate}
\item[Time Scale:] 5-10 years
\item[Convergence:]  Brain, HCI, Nanotechnology, Design
\item[Significant  Bottlenecks:] Being able to mimic the fine details of facial  expressions, universality in emotional expression.
\end{description}
 

\subsubsection{Artificial General Intelligence (AGI) rovers}
\begin{description}  \item[Application:] Applying \gls{AGI} in exploration rovers to be sent on  future mission to planets.
\item[Problem:] New environments  present different and unexpected challenges; obstacles, atmospheric  conditions, geological dangers all constitute a part of the unknown  circumstances that the rover might encounter. Making the right decision  is crucial to safely navigate and complete a successful mission. Current  rovers have no ability to predict or estimate danger, and have no  advanced decision making capabilities, thus making them vulnerable to  sudden or subtle changes in their surrounding, resulting in endangering  their mission and its outcomes.
\item[Opportunity:]  Advances in reverse engineering the brain and modeling how the mind  works, can give us the basis of how to implement a general approach of  problem solving mechanism that can be widely used in different  situations. If applied and integrated in space exploration robots, it  can facilitate its mission, making it more capable at gathering and  analyzing information, predicting potential beneficial areas of  exploration even if it was not originally planned in its mission.
\item[Grand Challenges:] Getting  There, Staying There, Robot Exploration
\item[Exponential Technologies:] Computing, Networking, AI,  Robotics
\item[Estimate of the Potential  Benefit:] - 
\item[Who is doing it:]
\hfill\begin{enumerate}
\item Numenta Project (Jeff Hawkins)
\item  OpenCog/Novamente (Ben  Goertzel)
\end{enumerate}
\item[Time Scale:] Significant  progress is being made towards a better understanding of how the mind  works, better models would probably emerge in the coming few years and  according to Ray Kurzweil we will achieve \gls{AGI} between 2015 and 2045.
\item[Convergence:] Brain, HCI,  Medical Scanning Instruments, Computation
\item[Significant  Bottlenecks:] The human brain complexity, information processing,  different theories and studies in the scientific community
\end{description}

 
\subsubsection{AI and Synthetic Biology for Space Related Human  Enhancement}
\label{enhancement}
\begin{description}
\item[Problem:]Space exploration, including life  on other planets, requires providing human life necessities during  potentially long space trips and long stays. 
 
 \item[Opportunity:] One approach would be to provide  the necessary living conditions and needs. However, another could be to  modify the required living conditions and needs by enhancing  human-beings, adjusting human physiology to space conditions, 
including:
\hfill\begin{enumerate}
\item adjusting for micro, or no  gravity
\item resistance to radiation
\item adjusting for a different  atmospheric blend and resistance to currently poisonous materials.
 \end{enumerate}
 \item[Exponential Technologies:] AI, Biotechnology
 \item[Grand Challenges:] Staying There, Getting There
 \item[Estimate of the Potential  Benefit:]
 Dramatically reducing the costs  and efforts required for space exploration and colonizing while  increasing survival chances.
 \item[Who is doing it:]
 \item[Time Scale:] Currently we know how to write around 1  million genes, sufficient to modify single cell organisms. As an  exponential technology, we expect to be 
able to  modify human DNA in 20 to 30 years.
 \item[Convergence:] Human enhancement techniques  could also help generating human-like robots.
 \item[Significant  Bottlenecks:]
 \hfill\begin{enumerate}
 \item There is a gap between the DNA  codes we can ``write'' and the real world requirements. AI could help bridge  that gap.
 \item Resistance to artificial life  forms and fear of the perils it 
 \end{enumerate}
  \end{description}



 
\subsubsection{Whole-brain simulation}
\label{brain-sim}
\begin{description}  \item[Application:] Whole Brain Simulation
 
\item[Problem:] Lack of knowledge  in brain-mind relationship
 
\item[Opportunity:]  As the computational power is exponentially increasing, simulation of  how the human brain works will become feasible.
 
\item[Exponential Technologies:]  Neural Technology and Computing
 
\item[Grand Challenges:] Staying  There. Space Exploration. Medicine in Space.
 
\item[Other Connected Ideas:]  Artificial General Intelligence 
 
\item[Estimate of the Potential  Benefit:]Cure neural and psychological diseases. Enhance human  performance. Develop AI based on the understanding of brain function.
 
\item[Who  is doing it:]
\hfill\begin{enumerate}
\item Blue Brain Project, IBM  Zurich, Switzerland, attempts to reverse-engineer the mammalian brain,  in order to understand brain function and dysfunction through detailed  simulations.
\item DARPA SyNAPSE, IBM  Almaden  Research Center, USA, investigates innovative approaches that enable  revolutionary advances in neuromorphic electronic devices that are  scalable to biological levels.  
\end{enumerate}
 
\item[Time  Scale:] 20 years
 
\item[Convergence:]  Artificial intelligence will be further advanced by fully understanding  the electrophysiology of human brain.
 
\item[Significant Bottlenecks:]  None of the existing invasive and noninvasive neural signal recording  techniques offer sufficient spatial and temporal resolutions to study  individual neuronal activity during behavior in real-time.
\end{description}

\subsubsection{Uploading intelligence}
 
\begin{description} \item[Application:] Uploading complete or partial human brains into space robots for exploration.
 
  \item[Problem:] Taking humans into deep space is risky, expensive and slow.
 
  \item[Opportunity:] To build exploration robots that can receive an ``uploaded'' human brain for science, exploration and other missions.
 
  \item[Exponential Technologies:] Neurotechnology, AI, Biotechnology,
Robotics, Networking.
 
  \item[Grand Challenges:] Robotic Exploration, Human Exploration, Getting
There, Staying There.
 
  \item[Other Connected Ideas:] High-throughput exploration through swarms of
sentient spacecraft; robotic exploration;
 
  \item[Estimate of the Potential Benefit:]-
 
  \item[Who is doing it:] The preliminary groundwork is being laid by IBM's Synapse and Blue Brain projects.
 
  \item[Time Scale:] 30 years
 
  \item[Convergence:] Bits, Neuroscience and Genetics, Robotics and AI
 
  \item[Significant Bottlenecks:] Technical---the capacity to scan brains functionally and the capacity to run them on space-rated hardware must be developed.
  \end{description}

 
\subsubsection{The Ultimate Fate of Space Exploration: Self-Replicating AGI Systems}

\begin{description}
\item[Application:]

Exploring more and more of the far reaches of space without any human
assistance.


\item[Problem:]

Typically today, all space missions require human intervention. This
greatly increases the cost and time involved. Additionally, all missions
must originate on Earth, which places a fundamental limit on the distance
from the Earth that can be explored in any given time. The optimal
situation would be exploration missions that do not require human
intervention, and which can replicate themselves \emph{while in space},
so that each subsequent generation of missions can explore further
and further into deep space.


\item[Opportunity:]

The ultimate manifestation of Artificial Intelligence is known as
Artificial General Intelligence (\glstext{AGI}). \glstext{AGI} systems are characterized
by being good at learning and innovating on their own, much like humans
do, and can become proficient at a wide array of tasks. This is in
stark contrast to currently available narrow AI systems, which are
only suitable for specifically engineered tasks. By combining \glstext{AGI}
with self-repicating systems, it will be possible to create systems
which explore the far reaches of the universe without any human intervention.
These systems will replicate themselves in the far reaches of space,
without ever having to return to Earth. They will be capable of learning
new things about Space, drawing their own conclusions, and making
novel decisions, without any human intervention.


\item[Exponential Technologies:]

Computing, Nanotechnology


\item[Grand Challenges:]

Robotic Exploration, Staying There


\item[Other Connected Ideas:]

This project is the logical extension of a more near-term project
that involves creating large numbers of semi-autonomous spacecraft
that are capable of exploring limited regions of space and reporting
back. By taking that project to the next logical extension, those
semi-autonomos spacecraft become fully autonomous. Finally, combining
this idea with self-replicating technology will allow the project
to grow exponentially without human intervention.


\item[Estimate of the Potential Benefit:]

This project is likely the culmination of space exploration. Self-replicating
robotic spacecraft will be able to cover far more reaches of space
than humans would ever be able to do, due to their exponentially growing
nature. It is estimated that the maximum amount of space will be explored
using a technique such as this one.


\item[Who is Doing It:]
\hfill
\begin{enumerate}
\item Newton Howard and Marvin Minsky are leading a project at MIT called
the MIT Mind Machine Project that has potential to create an intelligent
system powerful enough for this task\cite{mmp}.
\item Ben Goertzel is leading a non-profit open source project called OpenCog,
as well as a commercial AGI research venture called Novamente, both
of which have the potential to create an intelligent system powerful
enough for this task\cite{goertzel}.
\item Professor Gregory Chirikjian is working on research about self-replicating
robotic systems at Johns Hopkins University, and has successfully
prototyped several\cite{chirikjian}.
\item Chirikjian has worked with Suthakorn and Zhou to explore the idea
of self-replicating machines for space exploration as well\cite{chirikjian2}.
\end{enumerate}

\item[Time Scale:]

In order to embark on a mission such as this one, AGI must be available
alongside self-replicating machines. It is estimated that this will
not happen until 30+ years in the future.


\item[Convergence:]

This project lies neatly at the intersection between Artificial Intelligence,
computing power, miniaturization, nanotechnology, and self-replication.


\item[Significant Bottlenecks:]

It is currently unknown how any AGI system would function, or precisely
how any self-replicating machine would actually be deployed. Therefore,
an extensive amount of research and development would have to be performed
before this project could be commenced.

\end{description}



%%%ABOVE THIS POINT IS  SORTED%%%


