%NOTE: don't worry about extra spaces which may appear in copy+pasting to and from
%Etherpad. Remove them if they bother you, but otherwise TeX will remove them for you
%when the document is compiled to PDF.

\subsubsection{Getting There}
%(to be written by Dmitriy)

\subsubsection{Staying There}
%(to be written by Eric)



\subsubsection{Human Exploration}
%(to be written by Sarah Jane)

\subsubsection{Robotic Exploration}
%(to be written by Jan)

\subsubsection{Space Resources}
%(to be written by Mike)

\subsubsection{Space Science}
%(to be  written by Diva)

\subsubsection{Space Education/Evangelism}
%(written by Jason, edited by davidad)

A key underlying problem to the limit  of humanity's progress in space is that the majority of the human  species is uninterested in space. Activities in space today do not  inspire the awe that they once did: instead of watching astronauts walk  upon the surface of a celestial body for the first time, today, if we're  lucky, we get to watch them repair the toilet on the International  Space Station. Educational outreach is a good start, but we need to  connect humanity to space in a more natural way. Global warming is now  perceived as a threat so compelling that most people believe we need to  do something about it. But this threat is dwarfed by that of remaining  on this planet indefinitely.

NASA's  education program attempts to make space and the STEM (Science  Technology Engineering and Math) curriculum exciting for students, but  they do it without an understanding of the progress exponential  technologies will have on the space industry. Students today rarely  realize that when they grow older they will have a completely different  tool set with which to tackle space. We should teach them what the world  will be like in a decade or two and excite them on the possibilities  they can create.

In  all instances of educating the public on the importance of space, from  youth to the elderly, opening the space frontier is never presented as  an explicit need---which it is. Space is often considered to be  primarily beneficial as an engine for creating technological spin-offs,  but even if space exploration paid no technological dividends, it would  still be crucially important. People don't appreciate the voyage of  Christopher Columbus to the Americas primarily because of the new  sextant developed for the journey that was eventually spun off to the  private sector and used in merchant shipping.

It is human nature that we need to  "see it to believe it," and maybe this is what is holding back humanity  from understanding why space exploration is a necessity for survival.  Until ordinary people can experience space through their own eyes, the  possibilities that it holds will not be fully understood. What will  happen when the first child can see our home from space, or when the  first ballet dancer experiences weightlessness, or when a paraplegic is  given mobility again? Let's not just tell humanity about space, let's  give them access to space.


\subsubsection{Insuring Humanity}
%(to be  written by davidad)

