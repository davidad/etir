\documentclass[english]{article}
\usepackage[T1]{fontenc}
\usepackage[latin9]{inputenc}
\usepackage{babel}

\begin{document}

\section{Problem Space: Resources}

Human resource consumption increases as the population of the Earth
increases and as each member of the population uses more resources,
but the amount of resources available on the Earth remains fixed.
Fortunately, we as a species are not limited to gathering these resources
terrestrially. While we are facing a shortage of both raw materials\cite{gordon}
and energy here on the planet, both of these resrouces are bountiful
in our solar system and beyond.

This observation is by no means novel. Right now, though, at the dawn
of this new decade, we find ourselves sitting on the brink of an inflection
point with respect to the harvesting of space-based resources and
bringing them back to Earth. For the first time, as a result of the
convergence of several pivotal and game-changing exponentially accelerating
technologies, it will not only be economically viable to supply and
power Earth from space, it will be a necessity.


\subsection{Terrestrial Shortage of Valuable Materials That Are Abundant in Space}

There are currently not enough raw materials present on planet Earth
for the world's current population to live with the quality of life
of the modern developed world\cite{gordon}. This shortage clearly
poses a large problem for the sustained growth of humanity in the
upcoming decades, for not only is it desirable for every human being
to have access to the same level of technology as those living in
the developed world, but those in the developed world strive to advance
their technological. Among others, some of the most important of these
scarce materials are platinum group metals\cite{gerlach}.

As these scarce materials are mined and extracted from the Earth,
they become more and more scarce, and as they become more scarce the
cost of mining them and extracting them increases. Fortunately, virtually
all of these terrestrially scarce materials are available in Space
and can be found in near-earth asteroids. In the past, the cost of
mining these asteroids and returning their materials to the earth
has been prohibitavely expensive. However, in the upcoming five to
twenty years, the convergence of a number of exponentially accelerating
technologies will result in the cost of mining these asteroids decreasing
exponentially. With the cost of mining these materials from the earth
increasing with every passing year, and the cost of mining them from
space decreasing every year, there will be a point in the near future
where it is not only economically viable but economically imperative
to mine asteroids and return their contents to Earth\cite{gerlach}.

There are many advantages of mining near-earth objects for resources
over obtaining the resources from Earth, even aside from terrestrial
shortage. First of all, it is likely that a high percentage prospecting
missions will successfuly find highly valuable asteroids, due to already
well-established science for analyzing the composition of asteroidal
bodies from Earth. The high-grade ore found in asteroids will make
processing (extractive metallurgy) quite easy. No negotiations will
be needed with existing landowners. Additionally, there are no environmental
laws to be dealt with, and mining and waste disposal will not have
any potentially destructive effecst on the terrestrial ecosystem.
Asteroid mining systems will be highly scalable, flexible, and reusable\cite{sonter}.

Additionally, even if there were not a terrestrial shortage of materials
like platinum group metals, there is no doubt that if we had more
of them, it would usher in a new era of abundance on our planet. Imagine
the advances in technology and quality of life that could be made
if engineers could always use the ideal material for the job without
having to worry about price or availability.

Finally, the successful mining of asteroids will be essential to the
ultimate expansion of our species into the solar system and beyond.
Launching heavy objects from the ground in to space makes much less
sense than constructing those objects in space from available resources.
These resources are all available in near-Earth bodies.

Thus, the value of mining asteroids for resources is clear. There
are a number of sub-problems that must be solved in order to successfully
accomplish such a mission\cite{gerlach}. Each of these problems can
be solved elegantly with a combination of exponentially advancing
technologies.

\begin{itemize}
\item \emph{Remotely mining asteroids without the need for humans. }Advances
in AI and robotics will make this task cheaper and more efficient,
and cheaper as robotics and microprocessors become cheaper.
\item \emph{Data Collection. }Improved microprocessors, storage technology,
and sensors will drop the price and enable more and better data to
be collected. 
\item \emph{Mass and performance of spacecraft.} New materials such as carbon
nanofibers and advanced composites will enable lighter structures
to be as strong or stronger than steel.
\item \emph{Launch and Propulsion. }Advances in launch (particularly beam-powered
launch), solar technology, and energy storage will greatly increase
the efficiency while decreasing the cost, dramatically cutting the
costs of an asteroid mining mission.
\item \emph{Design}. Advances in software modeling and design will dramatically
increase the chance of success and decrease the cost of each mission.
\item \emph{Miniaturization. }The smaller a mining vehicle is, the more
can be launched per payload and the more fault-tolerant the entire
overall mission can be. Many technologies are getting smaller every
year, and this will dramatically affect the industry.
\end{itemize}

In sum, it is absolutely essential for humans to mine asteroids for
materials, since Earth is running out of resources, it will be essential
for building space-exploration machinery in place, it will usher in
a new era of abundance on Earth, and the cost of mining space-based
materials will soon be cheaper than mining those same materials terrestrially.
By leveraging a key set of exponentially advancing technologies, it
will for the first time be possible very shortly to mine asteroids
effectively, efficiently, and profitably.


\subsection{Energy and Power Available From Space}

{[}Jason Dunn has agreed to write this portion, and will submit it
separately in his own file{]}


\section{Selected Paragraphs on Exponential Technology Impacts}


\subsection{Resources in 10 Years: 3D Printing Cheap Small Mining Craft}


\subsubsection{Application}

Generating small, cheap, fairly reliable mining spacecraft for obtaining
valuable and essential resources from asteroids.


\subsubsection{Problem}

Current manufacturing processes are expensive and require a great
deal of manpower. Therefore, extreme care must be taken to make sure
that asteroid mining missions do not fail, which drives the cost up
even more. If mining spacecraft could be produced at a fraction of
the cost (and a fraction of the size), one mission (one payload) could
easily launch 100 or even 1,000 mining spacecraft, each to a different
asteroid. This would dramatically increase the chance of success for
the mission overall, as well as the fruits of each mission, while
simultaneously decreasing the cost of the mission as a whole.


\subsubsection{Opportunity}

Three-dimensional printing technology has already dramatically cut
the costs and time involved in manufacturing products in several industries.
As the field advances, 3D printers will be capable of printing with
more materials, cheaper, and faster. Shortly, these printers will
be capable of simply printing entire mining spacecraft. Effectively,
the impact will be equivalent or even better than instantaneously
erecting a mass-manufacturing plant for mining spacecraft for \$0.
As a result, the cost of producing small and effective mining spacecraft
will drop sharply.


\subsubsection{Exponential Technologies}

Nanotechnology, Computing


\subsubsection{Grand Challenges}

Robotic Exploration, Staying There, Resources


\subsubsection{Other Connected Ideas}

In order to make this idea work, it will also be necessary to combine
the 3D printing technologies with miniaturization, sufficient automation,
and robotics.


\subsubsection{Estimate of the Potential Benifit}

It is estimated that there are trillions of dollars worth of materials
available in near-Earth asteroids. Bringing these materials back successfully
would not only greatly benefit the mining entity financially, but
it would also unlock billions of people on the planet from being restricted
by the terrestrial shortage of the materials that are abundant in
asteroids.


\subsubsection{Who is Doing It}

\begin{enumerate}
\item Professor Hod Lipson at the Cornell Computational Synthesis Library
heads the Fab@Home project, which is an open source project that has
already resulted in the 3D printing of part of a working robot in
2009\cite{fabathome}.
\item Hod Lipson, along with Jordan B. Pollack, actually succesfully created
robots using 3D printing technology as a part of The Golem Project
at Brandeis University in the year 2000\cite{golem}.
\item Reshko, Mason, and Nourbakhsh have successfully used rapid-protoyping
technology to create small robots in the year 2000\cite{reshko}.
\item Won, DeLaruentis, and Marvodis at Rutgers University are also working
on rapidly prototyping robotic systems\cite{won}.
\end{enumerate}

\subsubsection{Time Scale}

We are just now seeing rapid prototyping technology over the last
few years that is capable of producing components of mining machines
already. We project that within the next 10 years it will be possible
to rapidly prototype an entire robotic mining system using 3D printing
technology..


\subsubsection{Convergence}

This opportunity lies at the intersection of advances in 3D printing,
design software, materials, microprocessor miniaturization, and robotics.


\subsubsection{Significant Bottlenecks}

Rapid prototyping and 3D printing is accelerating a fast clip. Since
the 3D printing of parts of a spacecraft is already highly feasible,
there are no significant bottlenecks that would have to be overcome
to achieve this goal.


\subsection{Space Exploration in 30+ Years: Remote Exponential Exploration of
Space Using a Set of Self-Replicationg AGI-Driven Systems}


\subsubsection{Application}

Exploring more and more of the far reaches of space without any human
assistance.


\subsubsection{Problem}

Typically today, all space missions require human intervention. This
greatly increases the cost and time involved. Additionally, all missions
must originate on Earth, which places a fundamental limit on the distance
from the Earth that can be explored in any given time. The optimal
situation would be exploration missions that do not require human
intervention, and which can replicate themselves \emph{while in space},
so that each subsequent generation of missions can explore further
and further into deep space.


\subsubsection{Opportunity}

The ultimate manifestation of Artificial Intelligence is known as
Artificial General Intelligence, or AGI. AGI systems are characterized
by being good at learning and innovating on their own, much like humans
do, and can become proficient at a wide array of tasks. This is in
stark contrast to currently available narrow AI systems, which are
only suitable for specifically engineered tasks. By combining AGI
with self-repicating systems, it will be possible to create systems
which explore the far reaches of the universe without any human intervention.
These systems will replicate themselves in the far reaches of space,
without ever having to return to Earth. They will be capable of learning
new things about Space, drawing their own conclusions, and making
novel decisions, without any human intervention.


\subsubsection{Exponential Technologies}

Computing, Nanotechnology


\subsubsection{Grand Challenges}

Robotic Exploration, Staying There


\subsubsection{Other Connected Ideas}

This project is the logical extension of a more near-term project
that involves creating large numbers of semi-autonomous spacecraft
that are capable of exploring limited regions of space and reporting
back. By taking that project to the next logical extension, those
semi-autonomos spacecraft become fully autonomous. Finally, combining
this idea with self-replicating technology will allow the project
to grow exponentially without human intervention.


\subsubsection{Estimate of the Potential Benifit}

This project is likely the culmination of space exploration. Self-replicating
robotic spacecraft will be able to cover far more reaches of space
than humans would ever be able to do, due to their exponentially growing
nature. It is estimated that the maximum amount of space will be explored
using a technique such as this one.


\subsubsection{Who is Doing It}

\begin{enumerate}
\item Newton Howard and Marvin Minsky are leading a project at MIT called
the MIT Mind Machine Project that has potential to create an intelligent
system powerful enough for this task\cite{mmp}.
\item Ben Goertzel is leading a non-profit open source project called OpenCog,
as well as a commercial AGI research venture called Novamente, both
of which have the potential to create an intelligent system powerful
enough for this task\cite{goertzel}.
\item Professor Gregory Chirikjian is working on research about self-replicating
robotic systems at Johns Hopkins University, and has successfully
prototyped several\cite{chirikjian}.
\item Chirikjian has worked with Suthakorn and Zhou to explore the idea
of self-replicating machines for space exploration as well\cite{chirikjian2}.
\end{enumerate}

\subsubsection{Time Scale}

In order to embark on a mission such as this one, AGI must be available
alongside self-replicating machines. It is estimated that this will
not happen until 30+ years in the future.


\subsubsection{Convergence}

This project lies neatly at the intersection between Artificial Intelligence,
computing power, miniaturization, nanotechnology, and self-replication. 


\subsubsection{Significant Bottlenecks}

It is currently unknown how any AGI system would function, or precisely
how any self-replicating machine would actually be deployed. Therefore,
an extensive amount of research and development would have to be performed
before this project could be commenced.


\subsection{Resources in 10 Years: Advances in AI and Robotics for Cheaper Mining
of Asteroids via Better Autonomy}


\subsubsection{Application}

Mining asteroids cheaply by letting the mining spacecraft handle most,
if not all of the decision-making.


\subsubsection{Problem}

Presently, all space missions require a certain degree of human control.
By cutting humans out of the loop, asteroid mining becomes much more
viable, not only because it is cheaper, but because thousands of missions
can be in operation concurrently.


\subsubsection{Opportunity}

By leveraging contemporary advances in AI and robotics, it will be
possible to create cheaper asteroid mining systems that perform better
on their own, without human intervention, than perhaps even human-guided
missions would. Since asteroid mining consists primarily of performing
mechanical tasks, contemporary AI and robotics greatly advance this
field.


\subsubsection{Exponential Technologies}

Computing, AI \& Robotics


\subsubsection{Grand Challenges}

Robotic Exploration, Staying There, Resources


\subsubsection{Other Connected Ideas}

This project is connected closely to the project of 3D-printing and
rapidly prototyping asteroid mining spacecraft.


\subsubsection{Estimate of the Potential Benifit}

The ability to launch a swarm of concurrent, cheap asteroid mining
missions without human intervention will allow the asteroid mining
problem to finally become economically viable.


\subsubsection{Who is Doing It}

\begin{enumerate}
\item Countless academic research groups and companies are working on producing
better autonomous robotic systems. For a good overview, see Bekey's
\emph{Autonomous Robots}, published in 2005, which surveys over 300
contemporary systems\cite{bekey}.
\item As an example for a future-looking autonomous robotics company, Willow
Garage, a company based in Menlo Park, CA, is working on developing
a standardized robotic operating system as well as a general-purpose
robot that could be eventaully applied for usage in asteroid mining.
They have just this year given 11 unites to leading research institutions
in robotics\cite{willow}.
\end{enumerate}

\subsubsection{Time Scale}

There are already many narrowly-intelligent artificial systems, as
well as many successful robotic applications here on Earth and in
space. It is estimated that in less than a decade, these technologies
will be applied commercially to asteroid mining.


\subsubsection{Convergence}

This opportunity lies at the convergence between Artificial Intelligence
and robotics.


\subsubsection{Significant Bottlenecks}

While there are many significant bottlenecks inherent in designing
better AI and better robotics, the AI and robotics systems that are
currently present on the Earth in 2010 are more than sufficient to
dramatically advance the asteroid mining endeavour.

\bibliographystyle{plain}
\bibliography{space}

\end{document}
